% Kelley Morse Theory of Sets and Classes

% We formalize the 

% Appendix: ELEMENTARY SET THEORY 

% of John L. Kelley 
% GENERAL TOPOLOGY 
% D. Van Nostrand Company Inc. 1955
%
% The appendix develops what is known as Kelley-Morse
% class theory (KM). 
% Kelley writes: "The system of axioms adopted is a variant
% of systems of Skolem and of A.P.Morse and owes much to
% the Hilbert-Bernays-von Neumann system as formulated
% by Gödel."

% This file covers the first 56 top level sections of the appendix.
% It uses SADs inbuilt class notion and mechanisms to model the
% classes of Kelley. We have built the class notion into 
% Naproche-SAD by replacing "set" by "class"

% This file checks in ~ 1 min on my laptop.

\documentclass[a4paper,draft]{amsproc}
\title{\textbf{Kelley}}

\date{}
\begin{document}

\theoremstyle{plain}
 \newtheorem{theorem}{Theorem}
 \newtheorem*{theorem*}{Theorem}
 \newtheorem{lemma}[theorem]{Theorem}
 \newtheorem*{lemma*}{Theorem}
 \newtheorem{proposition}[theorem]{Theorem}
 \newtheorem*{proposition*}{Theorem}
\theoremstyle{definition}
 \newtheorem{example}[theorem]{Example}
 \newtheorem*{example*}{Example}
 \newtheorem{definition}[theorem]{Definition}
 \newtheorem*{definition*}{Definition}
 \newtheorem{signature}[theorem]{Signature}
 \newtheorem*{signature*}{Signature}
\theoremstyle{remark}
 \newtheorem{remark[theorem]}{Remark}
 \newtheorem*{remark*}{Remark}
 \newtheorem{notation}[theorem]{Notation}
 \newtheorem*{notation*}{Notation}
\theoremstyle{axiom}
 \newtheorem{axiom}{Axiom}
 \newtheorem*{axiom*}{Axiom}
 \numberwithin{equation}{section}

\newenvironment{forthel}{}{}
\maketitle

\newcommand{cal}[1]{\mathcal{#1}}
\newcommand{domain}[1]{\textnormal{domain}}
\newcommand{\range}[1]{\textnormal{range}}

\begin{forthel}
[synonym element/-s]
\begin{signature}[ElmSort]
An element is a notion.
\end{signature}

%[synonym xobject/-s]
%\begin{signature}
%An xobject is a notion.
%\end{signature}

%[synonym class/-es]
%\begin{signature}
%A class is a notion.
%\end{signature}
\end{forthel}

\section{The Classification Axiom Scheme}
\begin{forthel}

%Let  x, y, z, r, s, t  denote classes.
%Let  b, c, d, e stand for xobjects.
Let  x, y, z, r, s, t, b, c, d, e  stand for sets.

Let  a neq b  stand for  a  !=  b .
Let  a in x  stand for  a  is an element of  x .

%\begin{axiom}
%Every element of  x  is an xobject.
%\end{axiom}

% Axiom I. Axiom of extent.
\begin{axiom}
For each  x, y   x = y  if and only if for each  z   z in x  if and only if  z in y .
\end{axiom}

% II Classification axiom-scheme corresponds to the way
% "classifications", i.e., abstraction terms are handled
% in Naproche-SAD

%[synonym set/-s]
%\begin{definition}
% A set is a class that is an element of some class.
%\end{definition}

% Definition 1
%\begin{definition} 
% x  is a set iff for some  y   x in y .
%\end{definition}
\end{forthel}

\section{ELEMENTARY ALGEBRA OF CLASSES}
\begin{forthel}

% Definition 2
%Let u denote an object.
\begin{definition} 
 x cup y = { set u | u in x  or  u in y } .
\end{definition}

% Definition 3
\begin{definition} 
 x cap y = { set u | u in x  and  u in y } .
\end{definition}

Let the union of  x  and  y  stand for  x cup y .
Let the intersection of  x  and  y  stand for  x cap y .

% Theorem 4
\begin{theorem}
 (z in x cup y  iff  z in x  or  z in y) 
and  (z in x cap y  iff  z in x  and  z in y) .
\end{theorem}

% Theorem 5
\begin{theorem}
 x cup x = x  and  x cap x = x .
\end{theorem}

% Theorem 6
\begin{theorem}
 x cup y = y cup x  and  x cap y = y cap x .
\end{theorem}

% Theorem 7
\begin{theorem}
 (x cup y) cup z = x cup (y cup z)  
and  (x cap y) cap z = x cap (y cap z) .
\end{theorem}

% Theorem 8
\begin{theorem}
 x cap (y cup z) = (x cap y) cup (x cap z) 
and  x cup (y cap z) = (x cup y) cap (x cup z) .
\end{theorem}

% 9 Definition, as a parser directive.
% Definition 9
Let  a notin b  stand for  a  is not an element of  b .

% Definition 10
\begin{definition}
sim x = { set  y | y notin x} .
\end{definition}
Let the complement of  x  stand for  sim x .

% Theorem 11
\begin{theorem}
 sim (sim x) = x .
\end{theorem}

% Theorem 12 (De Morgan)
\begin{theorem}
 sim (x cup y) = (sim x) cap (sim y)  
and  sim (x cap y) = (sim x) cup (sim y) .
\end{theorem}

% Theorem 13
\begin{definition}  x sim y = x cap (sim y) .\end{definition}

% Theorem 14
\begin{theorem}
 x cap (y sim z) = (x cap y) sim z .
\end{theorem}

% Definition 15
\begin{definition}  0 = { set  x | x neq x} .\end{definition}
Let the void class stand for  0 .
Let zero stand for  0 .

% Theorem 16
\begin{theorem}
 x notin 0 .
\end{theorem}

% Theorem 17
\begin{theorem}
 0 cup x = x  and  0 cap x = 0 .
\end{theorem}

% Definition 18
\begin{definition}
	 cal{U} = { set  x | x = x} .
\end{definition}
Let the universe stand for  cal{U} .

% Theorem 19
\begin{theorem}
 x in cal{U}  if and only if  x  is a set.
\end{theorem}

% Theorem 20
\begin{theorem}
 x cup cal{U} = cal{U}  and  x cap cal{U} = x .
\end{theorem}

% Theorem 21
\begin{theorem}
 sim 0 = cal{U}  and  sim cal{U} = 0 .
\end{theorem}

% Definition 22
\begin{definition}  bigcap x = 
{ set  z |  for each  y  if  y in x  then  z in y} .\end{definition}

% Definition 23
\begin{definition}  bigcup x = 
{ set  z |  for some  y   (z in y  and  y in x)} .\end{definition}

Let the intersection of  x  stand for  bigcap x .
Let the union of  x  stand for  bigcup x .

% Theorem 24
\begin{theorem}
 bigcap 0 = cal{U}  and  bigcup 0 = 0 .
\end{theorem}


%\begin{signature}
%A subclass is a notion.
%\end{signature}


[synonym subclass/-es]
% Definition 25
\begin{definition}  
%x  is a subclass of  y  iff  for each  z  if  z in x  then  z in y .
A subclass of y is a set x such that each element of x is an
element of y. 
\end{definition}

Let  x subset y  stand for  x  is a subclass of  y .
Let  x  is contained in  y  stand for  x subset y .

% Unnumbered
\begin{lemma}
 0 subset 0  and  0 notin 0 .
\end{lemma}

% Theorem 26
\begin{theorem}
 0 subset x  and  x subset cal{U} .
\end{theorem}

% Theorem 27
\begin{theorem}
 x = y  iff  x subset y  and  y subset x .
\end{theorem}

% Theorem 28
\begin{theorem}
If  x subset y  and  y subset z  then  x subset z .
\end{theorem}

% Theorem 29
\begin{theorem}
 x subset y  iff  x cup y = y .
\end{theorem}

% Theorem 30
\begin{theorem}
 x subset y  iff  x cap y = x .
\end{theorem}

% Theorem 31
\begin{theorem}
If  x subset y  then  bigcup x subset bigcup y 
and  bigcap y subset bigcap x .
\end{theorem}

% Theorem 32
\begin{theorem}
If  x in y  then  x subset bigcup y  
and  bigcap y subset x .
\end{theorem}
\end{forthel}

\section{EXISTENCE OF SETS}
\begin{forthel}

% Axiom of subsets.
\begin{axiom}
If  x  is a set then there is a set  y  such that for each
 z  if  z subset x  then  z in y .
\end{axiom}

% This axiom is a kind of powerclass axiom, where the powerclass
% also has all subCLASSES as elements.

% Theorem 33
\begin{theorem}
If  x  is a set and  z subset x  then  z  is a set.
\end{theorem}

% Theorem 34
\begin{theorem}
 0 = bigcap cal{U}  and  cal{U} = bigcup cal{U} .
\end{theorem}

% Theorem 35
\begin{theorem}
If  x neq 0  then  bigcap x  is a set.
\end{theorem}

% Definition 36
\begin{definition}  2^{x} = { set  y | y subset x} .\end{definition}

% Theorem 37
\begin{theorem}
 cal{U} = 2^{cal{U}} .
\end{theorem}

% Theorem 38
%[prove off]
\begin{theorem}
If  x  is a set then  2^{x}  is a set and for
each  y    y subset x  iff  y in 2^{x} .
\end{theorem}
%\begin{proof}
%Let  x  be a set.
%Take a set  y  such that for each  z  
%if  z subset x  then  z in y .
% 2^{x} subset y .
%\end{proof}
%[/prove]

% The Russell paradox.
% Unnumbered
\begin{definition}  cal{R} = { set  x | x notin x} .\end{definition}

% Unnumbered
\begin{theorem}
 cal{R}  is not a set.
\end{theorem}

% Theorem 39
\begin{theorem}
 cal{U}  is not a set.
\end{theorem}

% Definition 40
\begin{definition}  \{x\} = { set  z |  if  x in cal{U}  then  z = x} .\end{definition}
Let the singleton of  x  stand for  \{x\} .

% Before We used <x> instead of {x} since {x} was an inbuilt 
% set notation   

% Theorem 41
\begin{theorem}
If  x  is a set then for each  y   y in \{x\}  iff  y = x .
\end{theorem}

% Theorem 42
\begin{theorem}
If  x  is a set then  \{x\}  is a set.
\end{theorem}
%\begin{proof}
%Let  x  be a set. Then  {x} subset 2^{x}  
%and  2^{x}  is a set.
%\end{proof}

% Theorem 43
\begin{theorem}
 \{x\} = cal{U}  iff  x  is not a set.
\end{theorem}

% Theorem 44a
\begin{theorem}
If  x  is a set then  bigcap \{x\} = x  
and  bigcup \{x\} = x .
\end{theorem}

% Theorem 44b
\begin{theorem}
If  x  is not a set then  bigcap \{x\} = 0 
and  bigcup \{x\} = cal{U} .
\end{theorem}

% Axiom IV. 
\begin{axiom}
If  x  is a set and  y  is a set then  x cup y  is a set.
\end{axiom}

% Definition 45
\begin{definition}  \{x,y\} = \{x\} cup \{y\} .\end{definition}
Let the unordered pair of  x  and  y  stand for  \{x,y\} .

% The following has been a problem before:
% We use <x,y> instead of {x y} because Naproche-SAD requires
% some symbolic or textual material between the variables
% x and y. We use {x;y} instead of {x,y} because the latter
% notion is an inbuilt set notation of Naproche-SAD.

% Theorem 46a
\begin{theorem}
If  x  is a set and  y  is a set 
then  \{x,y\}  is a set and  (z in \{x,y\}  iff  z=x  or  z=y ). 
\end{theorem}

% Theorem 46b
\begin{theorem}
 \{x,y\} = cal{U}  if and only if  x  is not a set or  y  is not a set.
\end{theorem}

% Theorem 47a
\begin{theorem}
If  x,y  are sets then  bigcap \{x,y\} = x cap y 
and  bigcup \{x,y\} = x cup y .
\end{theorem}
%\begin{proof}
%Let  x,y  be sets.
% bigcup \{x,y\} subset x cup y .
% x cup y subset bigcup \{x,y\} .
%\end{proof}

% Theorem 47b
\begin{theorem}
If  x  is not a set or  y  is not a set then
 bigcap \{x,y\} = 0  and  bigcup \{x,y\} = cal{U} .
\end{theorem}
\end{forthel}

\section{ORDERED PAIRS: RELATIONS}
\begin{forthel}

% Definition 48
\begin{definition}  [x,y] = \{\{x\},\{x,y\}\} .\end{definition}
Let the ordered pair of  x  and  y  stand for  [x,y] .

% Theorem 49a
\begin{theorem}
 [x,y]  is a set if and only if  x  is a set and  y  is a set.
\end{theorem}

% Theorem 49b
\begin{theorem}
If  [x,y]  is not a set then  [x,y] = cal{U} .
\end{theorem}

% Theorem 50a
\begin{theorem}
If  x  and  y  are sets then 
   bigcup [x,y] = {x,y}  and
   bigcap [x,y] = {x}  and
   bigcup bigcap [x,y] = x  and
   bigcap bigcap [x,y] = x  and
   bigcup bigcup [x,y] = x cup y  and
   bigcap bigcup [x,y] = x cap y .
\end{theorem}

% Theorem 50b
\begin{theorem}
If  x  is not a set or  y  is not a set then
   bigcup bigcap [x,y] = 0  and
   bigcap bigcap [x,y] = cal{U}  and
   bigcup bigcup [x,y] = cal{U}  and
   bigcap bigcup [x,y] = 0 .
\end{theorem}

% Definition 51
\begin{definition}  1^{st}  coord  z = bigcap bigcap z .\end{definition}

% Definition 52
\begin{definition}  2^{nd}  coord  z = (bigcap bigcup z) cup 
((bigcup bigcup z) sim bigcup bigcap z) .\end{definition} 
Let the first coordinate of  z  stand for  1^{st}  coord  z .
Let the second coordinate of  z  stand for 2^{nd}  coord  z .

% Theorem 53
\begin{theorem}
 2^{nd}  coord  cal{U} = cal{U} .
\end{theorem}

% Theorem 54a
\begin{theorem}
If  x  and  y  are sets 
then  1^{st}  coord  [x,y] = x  and  2^{nd}  coord  [x,y] = y .
\end{theorem}
%\begin{proof}
%Let  x  and  y  be sets.
% 2^{nd}  coord  [x,y] = (bigcap bigcup [x,y]) cup 
%((bigcup bigcup [x,y]) sim bigcup bigcap [x,y])
%= (x cap y) cup ((x cup y) sim x)
%= y .
%\end{proof}

% Theorem 54b
\begin{theorem}
If  x  is not a set or  y  is not a set then
 1^{st}  coord  [x,y] = cal{U}  and 
 2^{nd}  coord  [x,y] = cal{U} .
\end{theorem}

Let u,v stand for sets.

% Theorem 55
\begin{theorem}
If  x  and  y  are sets and  [x,y] = [u,v]  then
 x = u  and  y = v .
\end{theorem}

% We can interpret cal{U} to mean undefined.
% Then ( , ) produces a a set or undefined.
% We can instead extend the signature (our language)
% by an elementary symbol ( , ), satisfying standard axioms ... .
% Ideally, we would like ( , ) to be an "object" and
% not a set. Sets will also be objects.

[synonym relation/-s]
% Definition 56
\begin{definition} 
%A class  rr  is a relation if and only if for each element  z  of  rr 
%there is  x  and  y  such that  z = [x,y] .
A relation is a set r such that for each z if z in r then there exist objects x and y such that z = [x,y].
%every element of r is an ordered pair.
\end{definition}

Let  r, s, t  stand for relations.

% Definition 57
\begin{definition}
 r circ s = {[x,z] | x, z  are objects and there exists an object  b  
such that  [x,b] in s  and  [b,z] in r} . 
\end{definition}

% Theorem 58
\begin{theorem}
 (r circ s) circ t = r circ (s circ t) .
\end{theorem}
%\begin{proof}
% (r circ s) circ t subset r circ (s circ t)  and
% r circ (s circ t) subset (r circ s) circ t .
%\end{proof}

% Theorem 59a
\begin{theorem}
 r circ (s cup t) = (r circ s) cup (r circ t) .
\end{theorem}
%\begin{proof}
% r circ (s cup t) subset (r circ s) cup (r circ t) .
% (r circ s) cup (r circ t) subset r circ (s cup t) .
%\end{proof}

% Theorem 59b
\begin{theorem}
 r circ (s cap t) subset (r circ s) cap (r circ t) .
\end{theorem}

% Definition 60
\begin{definition}
 r^{-1} = {[b,a] | a, b  are objects and  [a,b] in r} .
\end{definition}
Let the relation inverse to  r  stand for  r^{-1} .

% Unnumbered
\begin{lemma}
 r^{-1}  is a relation.
\end{lemma}

% Theorem 61
\begin{theorem}
 (r^{-1})^{-1} = r .
\end{theorem}
%\begin{proof}
% r subset (r^{-1})^{-1} .
% (r^{-1})^{-1} subset r .
%\end{proof}

% Lemma 62a
\begin{lemma}
Assume  r subset s . Then  r^{-1} subset s^{-1} .
\end{lemma}

%[/prove]

% Lemma 62b
\begin{lemma}
 (r circ s)^{-1} subset (s^{-1}) circ (r^{-1}) .
%\end{lemma}
%Proof.
%Let  u in (r circ s)^{-1} .
%Take  c  and  a  such that  u = [c,a] .
%Take an object  b  such that ( [a,b] in s  and  [b,c] in r ).
%Indeed  [a,c] in r circ s .
% [b,a] in s^{-1}  and  [c,b] in r^{-1} .
%Then  [c,a] in (s^{-1}) circ (r^{-1}) .
%qed.
\end{lemma}

% Unnumbered
\begin{lemma}
 (s^{-1}) circ (r^{-1}) subset (r circ s)^{-1} .
Proof.
 ((s^{-1}) circ (r^{-1}))^{-1} subset ((r^{-1})^{-1}) circ ((s^{-1})^{-1})  (by 62b).
 ((s^{-1}) circ (r^{-1}))^{-1} subset r circ s  (by 61).
 (((s^{-1}) circ (r^{-1}))^{-1})^{-1} subset (r circ s)^{-1}  (by 62a).
 (s^{-1}) circ (r^{-1}) subset (r circ s)^{-1}  (by 61).
qed.
\end{lemma}

% Theorem 62
\begin{theorem}
 (r circ s)^{-1} = (s^{-1}) circ (r^{-1}) .
Proof.
 (r circ s)^{-1} subset (s^{-1}) circ (r^{-1}) .
 (s^{-1}) circ (r^{-1}) subset (r circ s)^{-1} .
qed.
\end{theorem}

% Functions

% Since "function" is predefined in SAD3, we use the word "map" instead.

%[/prove]

[synonym map/-s]
% Definition 63
\begin{definition}
%A map is a relation  f  such that for each  a, b, c 
%if  [a,b] in f  and  [a,c] in f  then  b = c .
A map is a relation f such that for each x,y,z if [x,y] in f and [x,z] in f then y=z.
\end{definition}

Let  f, g  stand for maps.

% Theorem 64
\begin{theorem}
If  f, g  are maps then  f circ g  is a map.
\end{theorem}

% Definition 65
\begin{definition}
 domain x = {object u |   there exists an object  v  such that  [u,v] in x} .
\end{definition}

% Definition 66
\begin{definition}
 range x = {object v |   there exists an object  u  such that  [u,v] in x} .
\end{definition}

% Theorem 67
\begin{theorem}
 domain cal{U} = cal{U}  and  range cal{U} = cal{U} .
Proof.
If  x in cal{U}  then  (x,0), (0,x) in cal{U} .
 x in domain cal{U}  and  x in range cal{U} .
qed.
\end{theorem}

% Signature 68
%Let  f  be a map. Let  u in domain f .
\begin{signature}
The value of  f  at  u  is an object  v  such that u in domain f and [u,v] in f .
\end{signature}
Let  f(u)  stand for the value of  f  at  u .

% Theorem 69
\begin{theorem}
If  x notin domain f  then  f(x) = cal{U} .
%If  x in domain f  then  f(x) in cal{U} .
Proof.
If  x notin domain f  then  {y: [x,y] in f} = 0  and  f(x) = cal{U}  (by 24).
%If  x in domain f  then  {y: (x,y) in f} neq 0  and  f(x)  is a set (by 35).
qed.
\end{theorem}

% can't solve yet
% Theorem 70
%\begin{theorem}
%f = {[u,f(u)] | u in domain f} .
%\end{theorem}

% Theorem 71
\begin{theorem}
If  domain f = domain g  and for each u such that u in domain f  f(u) = g(u) then  f = g .
Proof.
Let us show that  f subset g .
Let  w in f . 
% Take objects  u, v  such that  w=[u,v] .
%   u in domain f  and  v = f(u) .
% Then  u in domain g  and  v = g(u) .
Then  w in g . end.
Let us show that  g subset f .
Let  w in g .  
Take objects  u, v  such that  w=[u,v] .
 u in domain g  and  v = g(u) .
Then  u in domain f  and  v = f(u) .
Then  w in f . end.
qed.
\end{theorem}

% Axiom V: Axiom of substitution
\begin{axiom}
If f is a map and domain f is a set then range f is a set.
\end{axiom}

% Axiom VI: Axiom of amalgamation
\begin{axiom}
If x is a set then bigcup x is a set.
\end{axiom}

% Definition 72
\begin{definition}
x times y = {[u,v] | u in x and v in y}.
\end{definition}

Let the cartesian product of x and y denote x times y.

% Theorem 73
\begin{theorem}
If u and y are sets then \{u\} times y is a set.
\end{theorem}

% Theorem 74
\begin{theorem}
If x and y are sets then x times y is a set.
\end{theorem}

% Theorem 75
\begin{theorem}
If f is a map and domain f is a set then f is a set.
Proof.
f subset (domain f) times (range f).
qed.
\end{theorem}

% Definition 76
\begin{definition}
y^x = {map f | domain f = x and range f subset y}.
\end{definition}

% Theorem 77
\begin{theorem}
If x and y are sets then y^x is a set.
\end{theorem}

% Definition 78
\begin{definition}
f is on x iff f is a map and x = domain f.
\end{definition}

% Definition 79
\begin{definition}
f is to y iff f is a map and range f subset y.
\end{definition}

% Definition 80
\begin{definition}
f is onto y iff f is a map and range f = y.
\end{definition}

\end{forthel}

\section{WELL ORDERING}
\begin{forthel}

Let r stand for a relation.

% Definition 81
% x \sim_{r} y
Let x (r) y stand for [x,y] in r.

% Definition 82
\begin{definition}
r connects x iff for all elements u,v of x u (r) v or v (r) u or v = u.
\end{definition}

% Definition 83
\begin{definition}
r is transitive in x iff for all elements u,v,w of x if u (r) v and v (r) w then u (r) w.
\end{definition}

% Definition 84
\begin{definition}
r is asymmetric in x iff for all elements u,v of x if u (r) v then not v (r) u.
\end{definition}

% Definition 85
% x neq y iff it is false that x = y.

% Definition 86
\begin{definition}
z minimizes x in r iff z in x. %and for each y in x not y (r) a.
\end{definition}

[synonym wellorder/-s]
% Definition 87
\begin{definition}
r wellorders x iff r connects x and for each subclass y of x such that y neq 0
there exists an object z such that z minimizes y in r.
\end{definition}

% Theorem 88a
\begin{theorem}
If r wellorders x then r is asymmetric in x.
Proof.
Assume r wellorders x.
Let u \in x and v \in x.
Assume u (r) v and v (r) u.
Then <u,v> \subset x.
Take an object z such that z minimizes <u,v> in r.
z = u or z = v.
Then not v (r) u or not u (r) v.
Contradiction. 
Then r is asymmetric in x.
qed.
\end{theorem}

% Theorem 88b
\begin{theorem}
If r wellorders x then r is transitive in x.
Proof.
Assume r wellorders x.
Assume that r is not transitive in x.
Take elements u,v,w of x such that 
u (r) v and v (r) w and not u (r) w.
u \neq w. Indeed r is asymmetric in x.
w (r) u.
There is no object a such that a minimizes (<u> \cup <v>) \cup <w> in r.
(<u> \cup <v>) \cup <w> \subset x.
Take an object b such that b minimizes (<u> \cup <v>) \cup <w> in r.
Contradiction.
qed.
\end{theorem}

[synonym section/-s]
% Definition 89
\begin{definition}
A section of x in r is a set y such that y subset x and r wellorders x
and for each objects u,v such that u in x and v in y and u (r) v u in y.
\end{definition}

% Theorem 90
\begin{theorem}
If z neq 0 and every element of z is a section of x in r then bigcup z and bigcap z are sections of x in r.
\end{theorem}

% Theorem 91
\begin{theorem}
If y is a section of x in r and y neq x then
y = {object u | u in x and u (r) v} for some element v of x.
Proof.
Let y be a section of x in r and y neq x.
Take an object v such that v minimizes x (r) y.
qed.
\end{theorem}

% Theorem 92
\begin{theorem}
If x and y are sections of z in r then x subset y or y subset x.
\end{theorem}

[synonym orderpreserve/-s]
% Definition 93
\begin{definition}
f orderpreserves r and s iff r wellorders domain f and s wellorders range f
and f(u) (s) f(v) for all elements u,v of domain f such that u (r) v.
\end{definition}

% Theorem 94
\begin{theorem}
If x is a section of y in r and f orderpreserves r and r and f is on x and f is to y
then for each element u of x not f(u) (r) u.
\end{theorem}

% Definition 95
\begin{definition}
A oneonefunction is a map f such that f^{-1} is a map.
\end{definition}

% Unnumbered
\begin{lemma}
f is a oneonefunction iff for all elements x,y of domain f such that x neq y f(x) neq f(y).
\end{lemma}

% Theorem 96a
\begin{theorem}
If f orderpreserves r and s then f is a oneonefunction.
\end{theorem}

% Theorem 96b
\begin{theorem}
If f orderpreserves r and s then f^{-1} orderpreserves s and r.
\end{theorem}

% Theorem 97
\begin{theorem}
If f and g orderpreserve r and s and domain f,domain g are sections of x in r
and range f,range g are sections of y in s then f subset g or g subset f.
\end{theorem}

% Definition 98
\begin{definition}
f orderpreserves r and s in x and y iff r wellorders x and s wellorders y and
f orderpreserves r and s and domain f is a section of y in s.
\end{definition}

% Theorem 99
\begin{theorem}
If r wellorders x and s wellorders y then there exists a map f such that
f orderpreserves r and s in x and y and (domain f = x or range f = y).
\end{theorem}

% Theorem 100
\begin{theorem}
If r wellorders x and s wellorders y and x is a set and y is not a set
then there exists a %unique
 map f such that f orderpreserves r and s in x and y
and domain f = x.
\end{theorem}

\end{forthel}
\end{document}
