\documentclass[a4paper,draft]{amsproc}
\title{\textbf{Kelley}}

\date{}
\begin{document}

\theoremstyle{plain}
 \newtheorem{theorem}{Theorem}
 \newtheorem*{theorem*}{Theorem}
 \newtheorem{lemma}[theorem]{Theorem}
 \newtheorem*{lemma*}{Theorem}
 \newtheorem{proposition}[theorem]{Theorem}
 \newtheorem*{proposition*}{Theorem}
\theoremstyle{definition}
 \newtheorem{example}[theorem]{Example}
 \newtheorem*{example*}{Example}
 \newtheorem{definition}[theorem]{Definition}
 \newtheorem*{definition*}{Definition}
 \newtheorem{signature}[theorem]{Signature}
 \newtheorem*{signature*}{Signature}
\theoremstyle{remark}
 \newtheorem{remark[theorem]}{Remark}
 \newtheorem*{remark*}{Remark}
 \newtheorem{notation}[theorem]{Notation}
 \newtheorem*{notation*}{Notation}
\theoremstyle{axiom}
 \newtheorem{axiom}{Axiom}
 \newtheorem*{axiom*}{Axiom}
 \numberwithin{equation}{section}

\newenvironment{forthel}{}{}
\maketitle

\newcommand{\cal}[1]{\mathcal{#1}}
\newcommand{\domain}[1]{\textnormal{domain}}
\newcommand{\range}[1]{\textnormal{range}}

\begin{forthel}
[synonym element/-s]

\begin{signature}[ElmSort]
An element is a notion.
\end{signature}

\end{forthel}

\section{The Classification Axiom Scheme}
\begin{forthel}

Let $x, y, z, u, v, a, b, c, d, e$ stand for classes.

Let $a \neq b$ stand for $a != b$.
Let $a \in b$ stand for $a$ is an element of $b$.

% Axiom I. Axiom of extent.
\begin{axiom}[I] 
For each class $x, y$ $x = y$ if and only if for each $z$ $z \in x$ 
when and only when $z \in y$.
\end{axiom}

% II Classification axiom-scheme corresponds to the way
% "classifications", i.e., abstraction terms are handled
% in Naproche-SAD

[synonym set/-s]

% Definition 1
\begin{definition}[1]
a set is a class $x$ such that for some class $y$ $x \in y$.
%x is a set iff for some y x \in y.
\end{definition}

%\section{ELEMENTARY ALGEBRA OF CLASSES}

% Definition 2
\begin{definition}[2] 
$x \cup y = \{$class $z : z \in x or z \in y \}$.
\end{definition}

% Definition 3
\begin{definition}[3] 
$x \cap y = \{$class $u : u \in x and u \in y \}$.
\end{definition}

Let the union of x and y stand for x $\cup$ y.
Let the intersection of x and y stand for x $\cap$ y.

% Theorem 4
\begin{theorem}[4]
($z \in x \cup y$ iff $z \in x$ or $z \in y$)
and ($z \in x \cap y$ iff $z \in x$ and $z \in y$).
\end{theorem}

% Theorem 5
\begin{theorem}[5]
$x \cup x = x$ and $x \cap x = x$.
\end{theorem}

% Theorem 6
\begin{theorem}[6]
$x \cup y$ = $y \cup x$ and $x \cap y = y \cap x$.
\end{theorem}

% Theorem 7
\begin{theorem}[7]
$(x \cup y) \cup z = x \cup (y \cup z)$ 
and $(x \cap y) \cap z = x \cap (y \cap z)$.
\end{theorem}

% Theorem 8
\begin{theorem}[8]
$x \cap (y \cup z) = (x \cap y) \cup (x \cap z)$
and $x \cup (y \cap z) = (x \cup y) \cap (x \cup z)$.
\end{theorem}

% 9 Definition, as a parser directive.
% Definition 9
Let $a \notin b$ stand for $a$ is not an element of $b$.

% Definition 10
\begin{definition}[10] 
$\sim x = \{$class $y : y \notin x\}$.
\end{definition}
Let the complement of x stand for $\sim x$.

% Theorem 11
\begin{theorem}[11]
$\sim (\sim x) = x$.
\end{theorem}

% Theorem 12 (De Morgan)
\begin{theorem}[12]
$\sim (x \cup y) = (\sim x) \cap (\sim y)$ 
and $\sim (x \cap y) = (\sim x) \cup (\sim y)$.
\end{theorem}

% Theorem 13
\begin{definition}[13] $x \sim y = x \cap (\sim y)$.\end{definition}

% Theorem 14
\begin{theorem}[14]
$x \cap (y \sim z) = (x \cap y) \sim z$.
\end{theorem}

% Definition 15
\begin{definition}[15] 
$0 = \{$class $x : x \neq x\}$.
\end{definition}
Let the void class stand for 0.
Let zero stand for 0.

% Theorem 16
\begin{theorem}[16]
$x \notin 0$.
\end{theorem}

% Theorem 17
\begin{theorem}[17]
$0 \cup x = x$ and $0 \cap x = 0$.
\end{theorem}

% Definition 18
\begin{definition}[18]
$\cal{U} = \{$class $x : x = x\}$.
\end{definition}
Let the universe stand for $\cal{U}$.

% Theorem 19
\begin{theorem}[19]
$x \in \cal{U}$ if and only if $x$ is a set.
\end{theorem}

% Theorem 20
\begin{theorem}[20]
$x \cup \cal{U} = \cal{U}$ and $x \cap \cal{U} = x$.
\end{theorem}

% Theorem 21
\begin{theorem}[21]
$\sim 0 = \cal{U}$ and $\sim \cal{U} = 0$.
\end{theorem}

% Definition 22
\begin{definition}[22]
$\bigcap x = \{$class $z$ : for each $y$ if $y \in x$ then $z \in y$\}.
\end{definition}

% Definition 23
\begin{definition}[23]
$\bigcup x = \{$class $z$ : for some $y (z \in y$ and $y \in x)$\}.
\end{definition}

Let the intersection of $x$ stand for $\bigcap x$.
Let the union of $x$ stand for $\bigcup x$.

% Theorem 24
\begin{theorem}[24]
$\bigcap 0 = \cal{U}$ and $\bigcup 0 = 0$.
\end{theorem}

% Definition 25
\begin{definition}[25]
A subclass of $y$ is a class $x$ such that for each class $z$ if $z \in x$ then $z \in y$.
\end{definition}

Let $x \subset y$ stand for $x$ is a subclass of $y$.
Let $x$ is contained in $y$ stand for $x \subset y$.

% Unnumbered
\begin{lemma}
$0 \subset 0$ and $0 \notin 0$.
\end{lemma}

% Theorem 26
\begin{theorem}[26]
$0 \subset x$ and $x \subset \cal{U}$.
\end{theorem}

% Theorem 27
\begin{theorem}[27]
$x = y$ iff $x \subset y$ and $y \subset x$.
\end{theorem}

% Theorem 28
\begin{theorem}[28]
If $x \subset y$ and $y \subset z$ then $x \subset z$.
\end{theorem}

% Theorem 29
\begin{theorem}[29]
$x \subset y$ iff $x \cup y = y$.
\end{theorem}

% Theorem 30
\begin{theorem}[30]
$x \subset y$ iff $x \cap y = x$.
\end{theorem}

% Theorem 31
\begin{theorem}[31]
If $x \subset y$ then $\bigcup x \subset \bigcup y$
and $\bigcap y \subset \bigcap x$.
\end{theorem}

% Theorem 32
\begin{theorem}[32]
If $x \in y$ then $x \subset \bigcup y$ 
and $\bigcap y \subset x$.
\end{theorem}

\end{forthel}

\section{EXISTENCE OF SETS}
\begin{forthel}

% Axiom of subsets.
\begin{axiom}[II]
If $x$ is a set then there is a set $y$ such that for each
$z$ if $z \subset x$ then $z \in y$.
\end{axiom}

% This axiom is a kind of powerclass axiom, where the powerclass
% also has all subCLASSES as elements.

% Theorem 33
\begin{theorem}[33]
If $x$ is a set and $z \subset x$ then $z$ is a set.
\end{theorem}

% Theorem 34
\begin{theorem}[34]
$0 = \bigcap \cal{U}$ and $\cal{U} = \bigcup \cal{U}$.
\end{theorem}

% Theorem 35
\begin{theorem}[35]
If $x \neq 0$ then $\bigcap x$ is a set.
\end{theorem}

% Definition 36
\begin{definition}[36]
$2^{x} = \{$class $y : y \subset x\}$.
\end{definition}

% Theorem 37
\begin{theorem}[37]
$\cal{U} = 2^{\cal{U}}$.
\end{theorem}

% Theorem 38
\begin{theorem}[38]
If $x$ is a set then $2^{x}$ is a set and for each $y$  $y \subset x$ iff $y \in 2^{x}$.
\end{theorem}
\begin{proof} Let $x$ be a set.
Take a set $y$ such that for each $z$ 
if $z \subset x$ then $z \in y$.
$2^{x} \subset y$.
\end{proof}

[prove off]
% The Russell paradox.
% Unnumbered
\begin{definition}[Russell Paradox]
$\cal{R} = \{$class $x : x \notin x\}$.
\end{definition}

% Unnumbered
\begin{theorem}
$\cal{R}$ is not a set.
\end{theorem}
\begin{proof}
Proof by contradiction.
Assume that $\cal{R}$ is a set.
If $\cal{R} \notin \cal{R}$ then $\cal{R} \in \cal{R}$.
Contradiction.
\end{proof}

% Theorem 39
\begin{theorem}[39]
$\cal{U}$ is not a set.
\end{theorem}
\begin{proof}
Proof by contradiction.
Assume that $\cal{U}$ is a set.
Then $\cal{R}$ is a set.
Contradiction.
\end{proof}
[/prove]

% Definition 40
\begin{definition}[40]
$<x> = \{$class $z$ : if $x \in \cal{U}$ then $z = x$\}.
\end{definition}
Let the singleton of $x$ stand for $<x>$.

% Before We used <x> instead of {x} since {x} was an inbuilt 
% set notation   

% Theorem 41
\begin{theorem}[41]
If $x$ is a set then for each class $y$ $y \in <x>$ iff $y = x$.
\end{theorem}

% Theorem 42
\begin{theorem}[42]
If $x$ is a set then $<x>$ is a set.
\end{theorem}
\begin{proof} Let $x$ be a set. Then $<x> \subset 2^{x}$ 
and $2^{x}$ is a set.
\end{proof}

% Theorem 43
\begin{theorem}[43]
$<x> = \cal{U}$ if and only if $x$ is not a set.
\end{theorem}

% Theorem 44a
\begin{theorem}[44a]
If $x$ is a set then $\bigcap <x> = x$ 
and $\bigcup <x> = x$.
\end{theorem}

% Theorem 44b
\begin{theorem}[44b]
If $x$ is not a set then $\bigcap <x> = 0$
and $\bigcup <x> = \cal{U}$.
\end{theorem}

% Axiom IV. 
\begin{axiom}[IV]
If $x$ is a set and $y$ is a set then $x \cup y$ is a set.
\end{axiom}

% Definition 45
\begin{definition}[45] $<x,y> = <x> \cup <y>$.\end{definition}
Let the unordered pair of $x$ and $y$ stand for $<x,y>$.

% The following has been a problem before:
% We use <x,y> instead of {x y} because Naproche-SAD requires
% some symbolic or textual material between the variables
% x and y. We use {x;y} instead of {x,y} because the latter
% notion is an inbuilt set notation of Naproche-SAD.

% Theorem 46a
\begin{theorem}[46a]
If $x$ is a set and y is a set 
then $<x,y>$ is a set and ($z \in <x,y>$ iff $z=x$ or $z=y$). 
\end{theorem}

% Theorem 46b
\begin{theorem}[46b]
$(<x,y>) = \cal{U}$ if and only if ($x$ is not a set or $y$ is not a set).
\end{theorem}

% Theorem 47a
\begin{theorem}[47a]
If x,y are sets then $\bigcap <x,y> = x \cap y$
and $\bigcup <x,y> = x \cup y$.
\end{theorem}
\begin{proof}
Let x,y be sets.
$\bigcup <x,y> \subset x \cup y$.
$x \cup y \subset \bigcup <x,y>$.
\end{proof}

% Theorem 47b
\begin{theorem}[47b]
If x is not a set or y is not a set then
$\bigcap <x,y> = 0$ and $\bigcup <x,y> = \cal{U}$.
\end{theorem}

%\section{ORDERED PAIRS: RELATIONS}

% Definition 48
\begin{definition}[48] $[x,y] = <<x>,<x,y>>$.\end{definition}
Let the ordered pair of $x$ and $y$ stand for $[x,y]$.

% Theorem 49a
\begin{theorem}[49a]
$[x,y]$ is a set if and only if $x$ is a set and $y$ is a set.
\end{theorem}

% Theorem 49b
\begin{theorem}[49b]
If $[x,y]$ is not a set then $[x,y] = \cal{U}$.
\end{theorem}

% Theorem 50a
\begin{theorem}[50a]
If $x$ and $y$ are sets then 
  $(\bigcup [x,y]) = (<x,y>)$ and
  $(\bigcap [x,y]) = <x>$ and
  $(\bigcup \bigcap [x,y]) = x$ and
  $(\bigcap \bigcap [x,y]) = x$ and
  $(\bigcup \bigcup [x,y]) = x \cup y$ and
  $(\bigcap \bigcup [x,y]) = x \cap y$.
\end{theorem}

% Theorem 50b
\begin{theorem}[50b]
If ($x$ is a class and $x$ is not a set) or ($y$ is a class and $y$ is not a set) then
  $\bigcup \bigcap [x,y] = 0$ and
  $\bigcap \bigcap [x,y] = \cal{U}$ and
  $\bigcup \bigcup [x,y] = \cal{U}$ and
  $\bigcap \bigcup [x,y] = 0$.
\end{theorem}

% Definition 51
\begin{definition}[51] $1^{st}$ coord $z = \bigcap \bigcap z$.\end{definition}

% Definition 52
\begin{definition}[52] $2^{nd}$ coord $z = (\bigcap \bigcup z) \cup 
((\bigcup \bigcup z) \sim \bigcup \bigcap z)$.\end{definition} 
Let the first coordinate of $z$ stand for $1^{st} coord z$.
Let the second coordinate of $z$ stand for $2^{nd} coord z$.

% Theorem 53
\begin{theorem}[53]
$2^{nd}$ coord $\cal{U} = \cal{U}$.
\end{theorem}

% Theorem 54a
\begin{theorem}[54a]
If $x$ and $y$ are sets 
then $1^{st}$ coord $[x,y] = x$ and $2^{nd}$ coord $[x,y] = y$.
\end{theorem}
\begin{proof}
Let $x$ and $y$ be sets.
$2^{nd}$ coord $[x,y] = (\bigcap \bigcup [x,y]) \cup 
((\bigcup \bigcup [x,y]) \sim \bigcup \bigcap [x,y])
= (x \cap y) \cup ((x \cup y) \sim x)
= y.$
\end{proof}

% Theorem 54b
\begin{theorem}[54b]
If ($x$ is a class and $x$ is not a set) or ($y$ is a class and $y$ is not a set) then
$1^{st}$ coord $[x,y]$ = $\cal{U}$ and 
$2^{nd}$ coord $[x,y]$ = $\cal{U}$.
\end{theorem}

% Theorem 55
\begin{theorem}[55]
If $x$ and $y$ are sets and $[x,y]$ = $[u,v]$ then
$x = u$ and $y = v$.
\end{theorem}

% We can interpret \cal{U} to mean undefined.
% Then ( , ) produces a a set or undefined.
% We can instead extend the signature (our language)
% by an elementary symbol ( , ), satisfying standard axioms ... .
% Ideally, we would like ( , ) to be an "object" and
% not a set. Sets will also be objects.

[synonym relation/-s]

% Definition 56
\begin{definition}[56] 
A relation is a class $r$ such that for each element $z$ of $r$ there exist $x$ and $y$ such that $z = [x,y]$.
\end{definition}

Let $r, s, t$ stand for relations.


% Definition 57
\begin{definition}[57]
$r \circ s = \{[x,z] | x,z$ are classes and there exists a class $y$ such that 
	$[x,y] \in s$ and $[y,z] \in r\}$. 
%r \circ s = \{class u | for some x,z u = [x,z] and for some y [x,y] \in s and [y,z] \in r\}.
\end{definition}

% Assisting
\begin{lemma}
$r \circ s$ is a relation.
\end{lemma}

% Assisting
\begin{lemma}
Assume $[a,c] \in r \circ s$. Then there is $b$ such that $[a,b] \in s$ and $[b,c] \in r$.
\end{lemma}
\begin{proof}
Take classes $k,b,m$ such that $[k,b] \in s$ and $[b,m] \in r$ and $[a,c] = [k,m]$.
Then $a = k$ and $c = m$.
\end{proof}

[prove off]
% Theorem 58
\begin{theorem}[58]
$(r \circ s) \circ t = r \circ (s \circ t)$.
\end{theorem}
\begin{proof}
Let us show that $(r \circ s) \circ t \subset r \circ (s \circ t)$.
  Let $z \in (r \circ s) \circ t$.
  Take classes $a,d$ such that $z = [a,d]$.
  Take classes $b,c$ such that $[a,b] \in t$ and $[b,c] \in s$ and $[c,d] \in r$.
  Then $[a,c] \in s \circ t$ and $z = [a,d]$ and $z \in r \circ (s \circ t)$.
end.
Let us show that $r \circ (s \circ t) \subset (r \circ s) \circ t$.
  Let $y \in r \circ (s \circ t)$.
  Take classes $m,q$ such that $y = [m,q]$.
  Take class $p$ such that $[m,p] \in s \circ t$ and $[p,q] \in r$.
  Take class $n$ such that $[m,n] \in t$ and $[n,p] \in s$.
  Then $[n,q] \in r \circ s$ and $y = [m,q]$ and $y \in (r \circ s) \circ t$.
end.
\end{proof}

% Theorem 59a
\begin{theorem}[59a]
$r \circ (s \cup t) = (r \circ s) \cup (r \circ t)$.
\end{theorem}
\begin{proof}
$r \circ (s \cup t) \subset (r \circ s) \cup (r \circ t)$.
$(r \circ s) \cup (r \circ t) \subset r \circ (s \cup t)$.
\end{proof}

% Theorem 59b
\begin{theorem}[59b]
$r \circ (s \cap t) \subset (r \circ s) \cap (r \circ t)$.
\end{theorem}
[/prove]

% Definition 60
\begin{definition}[60]
$r^{-1} = \{$class $s |$ for some $a,b$ $s = [b,a]$ and $[a,b] \in r\}$.
\end{definition}
Let the relation inverse to $r$ stand for $r^{-1}$.

% Unnumbered
\begin{lemma}
$r^{-1}$ is a relation.
\end{lemma}

%[prove off]
% Theorem 61
\begin{theorem}[61]
$(r^{-1})^{-1} = r$.
\end{theorem}
\begin{proof}
$r \subset (r^{-1})^{-1}$.
$(r^{-1})^{-1} \subset r$.
\end{proof}
[/prove]

% Lemma 62a
\begin{lemma}[62a]
Assume $r \subset s$. Then $r^{-1} \subset s^{-1}$.
\end{lemma}

%[prove off]
% Lemma 62b
\begin{lemma}[62b]
$(r \circ s)^{-1} \subset (s^{-1}) \circ (r^{-1})$.
\end{lemma}
%\begin{proof}
%Let u \in (r \circ s)^{-1}.
%Take c and a such that u = [c,a].
%Take an object b such that ([a,b] \in s and [b,c] \in r).
%Indeed [a,c] \in r \circ s.
%[b,a] \in s^{-1} and [c,b] \in r^{-1}.
%Then [c,a] \in (s^{-1}) \circ (r^{-1}).
%\end{proof}

% Unnumbered
\begin{lemma}
$(s^{-1}) \circ (r^{-1}) \subset (r \circ s)^{-1}$.
\end{lemma}
%\begin{proof}
%Let r,s be relations.
%((s^{-1}) \circ (r^{-1}))^{-1} \subset ((r^{-1})^{-1}) \circ ((s^{-1})^{-1}) (by 62b).
%((s^{-1}) \circ (r^{-1}))^{-1} \subset r \circ s (by 61).
%(((s^{-1}) \circ (r^{-1}))^{-1})^{-1} \subset (r \circ s)^{-1} (by 62a).
%(s^{-1}) \circ (r^{-1}) \subset (r \circ s)^{-1} (by 61).
%\end{proof}
[/prove]

% Theorem 62
\begin{theorem}[62]
$(r \circ s)^{-1} = (s^{-1}) \circ (r^{-1})$.
\end{theorem}
\begin{proof}
Let $r,s$ be relations.
$(r \circ s)^{-1} \subset (s^{-1}) \circ (r^{-1})$.
$(s^{-1}) \circ (r^{-1}) \subset (r \circ s)^{-1}$.
\end{proof}

\end{forthel}
\end{document}