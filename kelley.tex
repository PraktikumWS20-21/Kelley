% Kelley Morse Theory of Sets and Classes

% We formalize the 

% Appendix: ELEMENTARY SET THEORY 

% of John L. Kelley 
% GENERAL TOPOLOGY 
% D. Van Nostrand Company Inc. 1955
%
% The appendix develops what is known as Kelley-Morse
% class theory (KM). 
% Kelley writes: "The system of axioms adopted is a variant
% of systems of Skolem and of A.P.Morse and owes much to
% the Hilbert-Bernays-von Neumann system as formulated
% by Gödel."

% This file covers the first 56 top level sections of the appendix.
% It uses SADs inbuilt class notion and mechanisms to model the
% classes of Kelley. We have built the class notion into 
% Naproche-SAD by replacing "set" by "class"

% This file checks in ~ 1 min on my laptop.



\documentclass[a4paper,draft]{amsproc}
\title{\textbf{Kelley}}

\date{}
\begin{document}

\theoremstyle{plain}
 \newtheorem{theorem}{Theorem}
 \newtheorem*{theorem*}{Theorem}
 \newtheorem{lemma}[theorem]{Theorem}
 \newtheorem*{lemma*}{Theorem}
 \newtheorem{proposition}[theorem]{Theorem}
 \newtheorem*{proposition*}{Theorem}
\theoremstyle{definition}
 \newtheorem{example}[theorem]{Example}
 \newtheorem*{example*}{Example}
 \newtheorem{definition}[theorem]{Definition}
 \newtheorem*{definition*}{Definition}
 \newtheorem{signature}[theorem]{Signature}
 \newtheorem*{signature*}{Signature}
\theoremstyle{remark}
 \newtheorem{remark[theorem]}{Remark}
 \newtheorem*{remark*}{Remark}
 \newtheorem{notation}[theorem]{Notation}
 \newtheorem*{notation*}{Notation}
\theoremstyle{axiom}
 \newtheorem{axiom}{Axiom}
 \newtheorem*{axiom*}{Axiom}
 \numberwithin{equation}{section}

\newenvironment{forthel}{}{}
\maketitle

\newcommand{\cal}[1]{\mathcal{#1}}
\newcommand{\domain}[1]{\textnormal{domain}}
\newcommand{\range}[1]{\textnormal{range}}
%\newcommand{xxx}{\} % geht das?

\begin{forthel}
[synonym element/-s]

\begin{signature}[ElmSort]
An element is a notion.
\end{signature}

% section not yet supported?
%\section{The Classification Axiom Scheme}

Let $x, y, z, u, v, a, b, c, d, e$ stand for classes.

Let $a \neq b$ stand for $a != b$.
Let $a \in b$ stand for $a$ is an element of $b$.

% Axiom I. Axiom of extent.
\begin{axiom}[I] 
For each class $x, y$ $x = y$ if and only if for each $z$ $z \in x$ 
when and only when $z \in y$.
\end{axiom}

% II Classification axiom-scheme corresponds to the way
% "classifications", i.e., abstraction terms are handled
% in Naproche-SAD

[synonym set/-s]

% Definition 1
\begin{definition}[1]
a set is a class $x$ such that for some class $y$ $x \in y$.
%x is a set iff for some y x \in y.
\end{definition}

%\section{ELEMENTARY ALGEBRA OF CLASSES}

% Definition 2
\begin{definition}[2] 
$x \cup y = \{$class $z : z \in x or z \in y \}$.
\end{definition}

% Definition 3
\begin{definition}[3] 
$x \cap y = \{$class $u : u \in x and u \in y \}$.
\end{definition}

Let the union of x and y stand for x \cup y.
Let the intersection of x and y stand for x \cap y.

% Theorem 4
\begin{theorem}[4]
($z \in x \cup y$ iff $z \in x$ or $z \in y$)
and ($z \in x \cap y$ iff $z \in x$ and $z \in y$).
\end{theorem}

% Theorem 5
\begin{theorem}[5]
$x \cup x = x$ and $x \cap x = x$.
\end{theorem}

% Theorem 6
\begin{theorem}[6]
$x \cup y$ = $y \cup x$ and $x \cap y = y \cap x$.
\end{theorem}

% Theorem 7
\begin{theorem}[7]
$(x \cup y) \cup z = x \cup (y \cup z)$ 
and $(x \cap y) \cap z = x \cap (y \cap z)$.
\end{theorem}

% Theorem 8
\begin{theorem}[8]
$x \cap (y \cup z) = (x \cap y) \cup (x \cap z)$
and $x \cup (y \cap z) = (x \cup y) \cap (x \cup z)$.
\end{theorem}

% 9 Definition, as a parser directive.
% Definition 9
Let $a \notin b$ stand for $a$ is not an element of $b$.

% Definition 10
\begin{definition}[10] 
$\sim x = \{$class $y : y \notin x\}$.
\end{definition}
Let the complement of x stand for \sim x.

% Theorem 11
\begin{theorem}[11]
$\sim (\sim x) = x$.
\end{theorem}

% Theorem 12 (De Morgan)
\begin{theorem}[12]
\sim (x \cup y) = (\sim x) \cap (\sim y) 
and \sim (x \cap y) = (\sim x) \cup (\sim y).
\end{theorem}

% Theorem 13
\begin{definition}[13] x \sim y = x \cap (\sim y).\end{definition}

% Theorem 14
\begin{theorem}[14]
x \cap (y \sim z) = (x \cap y) \sim z.
\end{theorem}

% Definition 15
\begin{definition}[15] 
0 = \{class x : x \neq x\}.
\end{definition}
Let the void class stand for 0.
Let zero stand for 0.

% Theorem 16
\begin{theorem}[16]
x \notin 0.
\end{theorem}

% Theorem 17
\begin{theorem}[17]
0 \cup x = x and 0 \cap x = 0.
\end{theorem}

% Definition 18
\begin{definition}[18]
\cal{U} = \{class x : x = x\}.
\end{definition}
Let the universe stand for \cal{U}.

% Theorem 19
\begin{theorem}[19]
x \in \cal{U} if and only if x is a set.
\end{theorem}

% Theorem 20
\begin{theorem}[20]
x \cup \cal{U} = \cal{U} and x \cap \cal{U} = x.
\end{theorem}

% Theorem 21
\begin{theorem}[21]
\sim 0 = \cal{U} and \sim \cal{U} = 0.
\end{theorem}

% Definition 22
\begin{definition}[22]
\bigcap x = \{class z : for each y if y \in x then z \in y\}.
\end{definition}

% Definition 23
\begin{definition}[23]
\bigcup x = \{class z : for some y (z \in y and y \in x)\}.
\end{definition}

Let the intersection of x stand for \bigcap x.
Let the union of x stand for \bigcup x.

% Theorem 24
\begin{theorem}[24]
\bigcap 0 = \cal{U} and \bigcup 0 = 0.
\end{theorem}

% Definition 25
\begin{definition}[25]
a subclass of y is a class x such that for each class z if z \in x then z \in y.
\end{definition}

Let x \subset y stand for x is a subclass of y.
Let x is contained in y stand for x \subset y.

% Unnumbered
\begin{lemma}
0 \subset 0 and 0 \notin 0.
\end{lemma}

% Theorem 26
\begin{theorem}[26]
0 \subset x and x \subset \cal{U}.
\end{theorem}

% Theorem 27
\begin{theorem}[27]
x = y iff x \subset y and y \subset x.
\end{theorem}

% Theorem 28
\begin{theorem}[28]
If x \subset y and y \subset z then x \subset z.
\end{theorem}

% Theorem 29
\begin{theorem}[29]
x \subset y iff x \cup y = y.
\end{theorem}

% Theorem 30
\begin{theorem}[30]
x \subset y iff x \cap y = x.
\end{theorem}

% Theorem 31
\begin{theorem}[31]
If x \subset y then \bigcup x \subset \bigcup y
and \bigcap y \subset \bigcap x.
\end{theorem}

% Theorem 32
\begin{theorem}[32]
If x \in y then x \subset \bigcup y 
and \bigcap y \subset x.
\end{theorem}

\end{forthel}

\section{EXISTENCE OF SETS}
\begin{forthel}

% Axiom of subsets.
\begin{axiom}
If x is a set then there is a set y such that for each
z if z \subset x then z \in y.
\end{axiom}

% This axiom is a kind of powerclass axiom, where the powerclass
% also has all subCLASSES as elements.

% Theorem 33
\begin{theorem}[33]
If x is a set and z \subset x then z is a set.
\end{theorem}

% Theorem 34
\begin{theorem}[34]
0 = \bigcap \cal{U} and \cal{U} = \bigcup \cal{U}.
\end{theorem}

% Theorem 35
\begin{theorem}[35]
If x \neq 0 then \bigcap x is a set.
\end{theorem}

% Definition 36
\begin{definition}[36]
2^{x} = \{class y : y \subset x\}.
\end{definition}

% Theorem 37
\begin{theorem}[37]
\cal{U} = 2^{\cal{U}}.
\end{theorem}

% Theorem 38
\begin{theorem}[38]
If x is a set then 2^{x} is a set and for each y  y \subset x iff y \in 2^{x}.
\end{theorem}
\begin{proof} Let x be a set.
Take a set y such that for each z 
if z \subset x then z \in y.
2^{x} \subset y.
\end{proof}

% The Russell paradox.
% Unnumbered
%\begin{definition}
%\cal{R} = \{class x : x \notin x\}.
%\end{definition}

% Test
%\begin{lemma}
%Contradiction.
%\end{lemma}

% Unnumbered
%\begin{theorem}
%\cal{R} is not a set.
%\end{theorem}
%\begin{proof}
%Proof by contradiction.
%Assume that \cal{R} is a set.
%If \cal{R} \notin \cal{R} then \cal{R} \in \cal{R}.
%Contradiction.
%\end{proof}

% Theorem 39
%\begin{theorem}[39]
%\cal{U} is not a set.
%\end{theorem}
%\begin{proof}
%Proof by contradiction.
%Assume that \cal{U} is a set.
%Then \cal{R} is a set.
%Contradiction.
%\end{proof}

% Definition 40
\begin{definition}[40]
<x> = \{class z : if x \in \cal{U} then z = x\}.
\end{definition}
Let the singleton of x stand for <x>.

% Before We used <x> instead of {x} since {x} was an inbuilt 
% set notation   

% Theorem 41
\begin{theorem}[41]
If x is a set then for each class y y \in <x> iff y = x.
\end{theorem}

% Theorem 42
\begin{theorem}[42]
If x is a set then <x> is a set.
\end{theorem}
\begin{proof} Let x be a set. Then <x> \subset 2^{x} 
and 2^{x} is a set.
\end{proof}

% Theorem 43
\begin{theorem}[43]
<x> = \cal{U} if and only if x is not a set.
\end{theorem}

% Theorem 44a
\begin{theorem}[44a]
If x is a set then \bigcap <x> = x 
and \bigcup <x> = x.
\end{theorem}

% Theorem 44b
\begin{theorem}[44b]
If x is not a set then \bigcap <x> = 0
and \bigcup <x> = \cal{U}.
\end{theorem}

% Axiom IV. 
\begin{axiom}[IV]
If x is a set and y is a set then x \cup y is a set.
\end{axiom}

% Definition 45
\begin{definition}[45] <x,y> = <x> \cup <y>.\end{definition}
Let the unordered pair of x and y stand for <x,y>.

% The following has been a problem before:
% We use <x,y> instead of {x y} because Naproche-SAD requires
% some symbolic or textual material between the variables
% x and y. We use {x;y} instead of {x,y} because the latter
% notion is an inbuilt set notation of Naproche-SAD.

% Theorem 46a
\begin{theorem}[46a]
If x is a set and y is a set 
then <x,y> is a set and (z \in <x,y> iff z=x or z=y). 
\end{theorem}

% Theorem 46b
\begin{theorem}[46b]
(<x,y>) = \cal{U} if and only if (x is not a set or y is not a set).
\end{theorem}

% Theorem 47a
\begin{theorem}[47a]
If x,y are sets then \bigcap <x,y> = x \cap y
and \bigcup <x,y> = x \cup y.
\end{theorem}
\begin{proof}
Let x,y be sets.
\bigcup <x,y> \subset x \cup y.
x \cup y \subset \bigcup <x,y>.
\end{proof}

% Theorem 47b
\begin{theorem}[47b]
If x is not a set or y is not a set then
\bigcap <x,y> = 0 and \bigcup <x,y> = \cal{U}.
\end{theorem}

%\section{ORDERED PAIRS: RELATIONS}

% Definition 48
\begin{definition}[48] [x,y] = <<x>,<x,y>>.\end{definition}
Let the ordered pair of x and y stand for [x,y].

% Theorem 49a
\begin{theorem}[49a]
[x,y] is a set if and only if x is a set and y is a set.
\end{theorem}

% Theorem 49b
\begin{theorem}[49b]
If [x,y] is not a set then [x,y] = \cal{U}.
\end{theorem}

% Theorem 50a
\begin{theorem}[50a]
If x and y are sets then 
  (\bigcup [x,y]) = (<x,y>) and
  (\bigcap [x,y]) = <x> and
  (\bigcup \bigcap [x,y]) = x and
  (\bigcap \bigcap [x,y]) = x and
  (\bigcup \bigcup [x,y]) = x \cup y and
  (\bigcap \bigcup [x,y]) = x \cap y.
\end{theorem}

% Theorem 50b
\begin{theorem}[50b]
If (x is a class and x is not a set) or (y is a class and y is not a set) then
  \bigcup \bigcap [x,y] = 0 and
  \bigcap \bigcap [x,y] = \cal{U} and
  \bigcup \bigcup [x,y] = \cal{U} and
  \bigcap \bigcup [x,y] = 0.
\end{theorem}

% Definition 51
\begin{definition}[51] 1^{st} coord z = \bigcap \bigcap z.\end{definition}

% Definition 52
\begin{definition}[52] 2^{nd} coord z = (\bigcap \bigcup z) \cup 
((\bigcup \bigcup z) \sim \bigcup \bigcap z).\end{definition} 
Let the first coordinate of z stand for 1^{st} coord z.
Let the second coordinate of z stand for 2^{nd} coord z.

% Theorem 53
\begin{theorem}[53]
2^{nd} coord \cal{U} = \cal{U}.
\end{theorem}

% Theorem 54a
\begin{theorem}[54a]
If x and y are sets 
then 1^{st} coord [x,y] = x and 2^{nd} coord [x,y] = y.
\end{theorem}
\begin{proof}
Let x and y be sets.
2^{nd} coord [x,y] = (\bigcap \bigcup [x,y]) \cup 
((\bigcup \bigcup [x,y]) \sim \bigcup \bigcap [x,y])
= (x \cap y) \cup ((x \cup y) \sim x)
= y.
\end{proof}

% Theorem 54b
\begin{theorem}[54b]
If (x is a class and x is not a set) or (y is a class and y is not a set) then
1^{st} coord [x,y] = \cal{U} and 
2^{nd} coord [x,y] = \cal{U}.
\end{theorem}

% Theorem 55
\begin{theorem}[55]
If x and y are sets and [x,y] = [u,v] then
x = u and y = v.
\end{theorem}

% We can interpret \cal{U} to mean undefined.
% Then ( , ) produces a a set or undefined.
% We can instead extend the signature (our language)
% by an elementary symbol ( , ), satisfying standard axioms ... .
% Ideally, we would like ( , ) to be an "object" and
% not a set. Sets will also be objects.

[synonym relation/-s]

% Definition 56
\begin{definition}[56] 
A relation is a class r such that for each element z of r there exist x and y such that z = [x,y].
\end{definition}

Let r, s, t stand for relations.


% Definition 57
\begin{definition}[57]
r \circ s = \{[x,z] | x,z are classes and there exists a class y such that 
	$[x,y] \in s$ and $[y,z] \in r\}$. 
%r \circ s = \{class u | for some x,z u = [x,z] and for some y [x,y] \in s and [y,z] \in r\}.
\end{definition}

% Assisting
\begin{lemma}
r \circ s is a relation.
\end{lemma}

% Assisting
\begin{lemma}
Assume [a,c] \in r \circ s. Then there is b such that [a,b] \in s and [b,c] \in r.
\end{lemma}
\begin{proof}
Take classes k,b,m such that [k,b] \in s and [b,m] \in r and [a,c] = [k,m].
Then a = k and c = m.
\end{proof}

[prove off]
% Theorem 58
\begin{theorem}[58]
(r \circ s) \circ t = r \circ (s \circ t).
\end{theorem}
\begin{proof}
%Let r,s,t be relations.
%(r \circ s) \circ t \subset r \circ (s \circ t) and
%r \circ (s \circ t) \subset (r \circ s) \circ t.
%\begin{proof}
Let us show that (r \circ s) \circ t \subset r \circ (s \circ t).
  Let z \in (r \circ s) \circ t.
  Take classes a,d such that z = [a,d].
  Take classes b,c such that [a,b] \in t and [b,c] \in s and [c,d] \in r.
  Then [a,c] \in s \circ t and z = [a,d] and z \in r \circ (s \circ t).
%Then (r \circ s) \circ t \subset r \circ (s \circ t).
%\end{proof}
end.
%r \circ (s \circ t) \subset (r \circ s) \circ t. 
%\begin{proof}
Let us show that r \circ (s \circ t) \subset (r \circ s) \circ t.
  Let y \in r \circ (s \circ t).
  Take classes m,q such that y = [m,q].
  Take class p such that [m,p] \in s \circ t and [p,q] \in r.
  Take class n such that [m,n] \in t and [n,p] \in s.
  Then [n,q] \in r \circ s and y = [m,q] and y \in (r \circ s) \circ t.
%\end{proof}
end.
%Then r \circ (s \circ t) \subset (r \circ s) \circ t.
\end{proof}

% Theorem 59a
\begin{theorem}[59a]
r \circ (s \cup t) = (r \circ s) \cup (r \circ t).
\end{theorem}
\begin{proof}
r \circ (s \cup t) \subset (r \circ s) \cup (r \circ t).
(r \circ s) \cup (r \circ t) \subset r \circ (s \cup t).
\end{proof}

% Theorem 59b
\begin{theorem}[59b]
r \circ (s \cap t) \subset (r \circ s) \cap (r \circ t).
\end{theorem}
[/prove]

% Definition 60
\begin{definition}[60]
r^{-1} = \{class s | for some a,b s = [b,a] and [a,b] \in r\}.
\end{definition}
Let the relation inverse to r stand for r^{-1}.

% Unnumbered
\begin{lemma}
r^{-1} is a relation.
\end{lemma}

[prove off]
% Theorem 61
\begin{theorem}[61]
(r^{-1})^{-1} = r.
\end{theorem}
\begin{proof}
r \subset (r^{-1})^{-1}.
(r^{-1})^{-1} \subset r.
\end{proof}
[/prove]

% Lemma 62a
\begin{lemma}[62a]
Assume r \subset s. Then r^{-1} \subset s^{-1}.
\end{lemma}

[prove off]
% Lemma 62b
\begin{lemma}[62b]
(r \circ s)^{-1} \subset (s^{-1}) \circ (r^{-1}).
\end{lemma}
%\begin{proof}
%Let u \in (r \circ s)^{-1}.
%Take c and a such that u = [c,a].
%Take an object b such that ([a,b] \in s and [b,c] \in r).
%Indeed [a,c] \in r \circ s.
%[b,a] \in s^{-1} and [c,b] \in r^{-1}.
%Then [c,a] \in (s^{-1}) \circ (r^{-1}).
%\end{proof}

% Unnumbered
\begin{lemma}
(s^{-1}) \circ (r^{-1}) \subset (r \circ s)^{-1}.
\end{lemma}
%\begin{proof}
%Let r,s be relations.
%((s^{-1}) \circ (r^{-1}))^{-1} \subset ((r^{-1})^{-1}) \circ ((s^{-1})^{-1}) (by 62b).
%((s^{-1}) \circ (r^{-1}))^{-1} \subset r \circ s (by 61).
%(((s^{-1}) \circ (r^{-1}))^{-1})^{-1} \subset (r \circ s)^{-1} (by 62a).
%(s^{-1}) \circ (r^{-1}) \subset (r \circ s)^{-1} (by 61).
%\end{proof}
[/prove]

% Theorem 62
\begin{theorem}[62]
(r \circ s)^{-1} = (s^{-1}) \circ (r^{-1}).
\end{theorem}
\begin{proof}
Let r,s be relations.
(r \circ s)^{-1} \subset (s^{-1}) \circ (r^{-1}).
(s^{-1}) \circ (r^{-1}) \subset (r \circ s)^{-1}.
\end{proof}

% Functions

% Since "function" is predefined in SAD3, we use the word "map" instead.

[synonym map/-s]
% Definition 63
\begin{definition}[63]
A map is a relation f such that for each class a, b, c
if [a,b] \in f and [a,c] \in f then b = c.
\end{definition}

Let f, g stand for maps.

[prove off]
% Theorem 64
\begin{theorem}[64]
If f, g are maps then f \circ g is a map.
\end{theorem}
[/prove]

% Definition 65
\begin{definition}[65]
\domain x = \{object u | for some v [u,v] \in x\}.
\end{definition}

% Definition 66
\begin{definition}[66]
\range x = \{object v | for some u [u,v] \in x\}.
\end{definition}

% Test
\begin{lemma}
\cal{U} is a map.
\end{lemma}

% Theorem 67
\begin{theorem}[67]
\domain \cal{U} = \cal{U} and \range \cal{U} = \cal{U}.
\end{theorem}
\begin{proof}
Let x be a class. Let y be a void class.
If x \in \cal{U} then [x,y], [y,x] \in \cal{U}.
x \in \domain \cal{U} and x \in \range \cal{U}.
\end{proof}

%\begin{definition}
%Let f be a map. Let u \in \domain f.
%\begin{definition}
%subdomain u = \{class v : [u,v] \in f\}.
%The subvalue of f at u is a class setx such that setx = \{class v : [u,v] \in f\}.
%\end{definition}

% Definition 68
\begin{signature}[68]
Let f be a map. %Let u \in \domain f.
The value of f at u is a class v such that [u,v] \in f.
%The subvalue of f at u is a class setx such that setx = \{class v : [u,v] \in f\}.
%The value of f at u is the intersection of the subvalue of f at u.
\end{signature}
Let f(u) stand for the value of f at u.

[prove off]
% Theorem 69
\begin{theorem}[69a]
If x \notin \domain f then f(x) = \cal{U}.
\end{theorem}
%\begin{proof}
%If x \notin \domain f 
%then \{class y | [x,y] \in f\} = 0 and f(x) = \cal{U} (by 24).
%\end{proof}

\begin{theorem}[69b]
If x \in \domain f then f(x) \in \cal{U}.
\end{theorem}
%\begin{proof}
%If x \in \domain f 
%then \{class y | [x,y] \in f\} \neq 0    % \neq 0 nicht möglich?
%and f(x) is a set (by 35).
%\end{proof}


% Theorem 70
%\begin{theorem}[70]
%If f is a map then f = \{[x,y] : x,y are classes and y = f(x)\}.
%If f is a map then f = {[u,f(u)] : u is a class and u \in \domain f}.
%\end{theorem}

[prove off]
% Theorem 71
\begin{theorem}[71]
If \domain f = \domain g and for each element u of \domain f f(u) = g(u) then f = g.
\end{theorem}
%\begin{proof}
%Let us show that f \subset g.
%\begin{proof}
%Let w \in f. 
%Then w \in g. 
%\end{proof}
%%end.
%Let us show that g \subset f.
%\begin{proof}
%Let w \in g.  
%Take objects u, v such that w=[u,v].
%u \in \domain g and v = g(u).
%Then u \in \domain f and v = f(u).
%Then w \in f. 
%\end{proof} %end.
%\end{proof}
[/prove]

% Axiom of substitution
\begin{axiom}[V]
Let f be a map. If \domain f is a set then \range f is a set.
\end{axiom}

% Axiom of amalgamation
\begin{axiom}[VI]
If x is a set then \bigcup x is a set.
\end{axiom}

[prove off]
% Definition 72
\begin{definition}[72]
 x \times y = \{[u,v] : u,v are classes and u \in x and v \in y\}.
\end{definition}

% Theorem 73
\begin {theorem}[73]
If u,v are sets then <u> \times v is a set.
\end{theorem}

% Theorem 74
\begin{theorem}[74]
If x,y are sets then x \times y is a set.
\end{theorem}
%\begin{proof}
%Let f be a map such that \domain f = x and
%f(u) = <u> \times y for every element u of x.
%\range f is a set.
%\range f = \{class z : for some class u (u \in x and z = (<u> \times y))\}.
%Consequently \bigcup \range f = x \times y.
%\end{proof}
[/prove]

% Theorem 75
\begin{theorem}[75]
If f is a map and \domain f is a set 
then f is a set.\end{theorem}
%\begin{proof}
%f \subset (\domain f \times \range f).
%\end{proof}

% Definition 76
\begin{definition}[76]
y^{x} = \{map f | \domain f = x and \range f \subset y\}.
\end{definition}

% Theorem 77
\begin{theorem}[77]
If x,y are sets then y^{x} is a set.
\end{theorem}
%\begin{proof}
%Let f be map.
%If f \in y^{x} then f \subset x \times y.
%f \in 2^{x \times y} (by 38).
%2^{x \times y} is a set.
%y^{x} \subset 2^{x \times y}.
%y^{x} is a set.
%\end{proof}

% Definition 78
\begin{definition}[78]
f is on x if and only if f is a map and x = \domain f.
\end{definition}

% Definition 79
\begin{definition}[79]
f is to y if and only if f is a map such and \range f \subset y.
\end{definition}

% Definition 80
\begin{definition}[80]
f is onto y if and only if f is a map and \range f = y.
\end{definition}

\end{forthel}

\section{WELL ORDERING}
\begin{forthel}

Let r stand for a relation.

% Definition 81
% x \sim_{r} y
Let x (r) y stand for [x,y] \in r.

% Definition 82
\begin{definition}[82]
r connects x iff for all elements u,v of x u (r) v or v (r) u or v = u.
\end{definition}

% Definition 83
\begin{definition}[83]
r is transitive in x iff for all elements u,v,w of x if u (r) v and v (r) w then u (r) w.
\end{definition}

% Definition 84
\begin{definition}[84]
r is asymmetric in x iff for all elements u,v of x if u (r) v then not v (r) u.
\end{definition}

% Definition 85
% x neq y iff it is false that x = y.

Let $x$ denote a class.
\begin{definition}[86]
z minimizes x in r if and only if (z \in x and for each element y of x not (y (r) z)).
\end{definition}

[synonym wellorder/-s]
% Definition 87
\begin{definition}[87]
r wellorders x if and only if r connects x and for each subclass y of x such that y \neq 0
there exists a class z such that z minimizes y in r.
\end{definition}

% Theorem 88a
\begin{theorem}[88a]
If r wellorders x then r is asymmetric in x.
\end{theorem}
%\begin{proof}
%Assume r wellorders x.
%Let u \in x and v \in x.
%Assume u (r) v and v (r) u.
%Then <u,v> \subset x.
%Take an object z such that z minimizes <u,v> in r.
%z = u or z = v.
%Then not v (r) u or not u (r) v.
%Contradiction. 
%Then r is asymmetric in x.
%\end{proof}

% Theorem 88b
\begin{theorem}[88b]
If r wellorders x then r is transitive in x.
\end{theorem}
%\begin{proof}
%Assume r wellorders x.
%Assume that r is not transitive in x.
%Take elements u,v,w of x such that 
%u (r) v and v (r) w and not u (r) w.
%u \neq w. Indeed r is asymmetric in x.
%w (r) u.
%There is no object a such that a minimizes (<u> \cup <v>) \cup <w> in r.
%(<u> \cup <v>) \cup <w> \subset x.
%Take an object b such that b minimizes (<u> \cup <v>) \cup <w> in r.
%Contradiction.
%\end{proof}

[synonym section/-s]
% Definition 89
\begin{definition}[89]
A section of x in r is a class y such that y subset x and r wellorders x
and for each classes u,v such that u \in x and v \in y and u (r) v u \in y.
\end{definition}

% Theorem 90
\begin{theorem}[90]
If z \neq 0 and every element of z is a section of x in r then \bigcup z and \bigcap z are sections of x in r.
\end{theorem}

% Theorem 91
\begin{theorem}[91]
If y is a section of x in r and y \neq x then there exists a class v such that v in x and 
y = \{class u | u \in x and u (r) v\}.
\end{theorem}
%\begin{proof}
%Let y be a section of x in r and y \neq x.
%Take an object v such that v minimizes x in r and x (r) y.
%\end{proof}

% Theorem 92
\begin{theorem}[92]
If x and y are sections of z in r then x \subset y or y \subset x.
\end{theorem}

[synonym orderpreserve/-s]
% Definition 93
\begin{definition}[93]
f orderpreserves r and s iff r wellorders \domain f and s wellorders \range f
and f(u) (s) f(v) for all elements u,v of \domain f such that u (r) v.
\end{definition}

% Theorem 94
\begin{theorem}[94]
If x is a section of y in r and f orderpreserves r and r and f is on x and f is to y
then for each element u of x not f(u) (r) u.
\end{theorem}

% Definition 95
\begin{definition}[95]
A oneonefunction is a map f such that f^{-1} is a map.
\end{definition}

% Theorem 96a
\begin{theorem}[96a]
If f orderpreserves r and s then f is a oneonefunction.
\end{theorem}

% Theorem 96b
\begin{theorem}[96b]
If f orderpreserves r and s then f^{-1} orderpreserves s and r.
\end{theorem}

% Theorem 97
\begin{theorem}[97]
If f and g orderpreserve r and s and \domain f,\domain g are sections of x in r
and \range f,\range g are sections of y in s then f \subset g or g \subset f.
\end{theorem}

% Definition 98
\begin{definition}[98]
f orderpreserves r and s in x and y iff r wellorders x and s wellorders y and
f orderpreserves r and s and \domain f is a section of y in s.
\end{definition}

% Theorem 99
\begin{theorem}[99]
If r wellorders x and s wellorders y then there exists a map f such that
f orderpreserves r and s in x and y and (\domain f = x or \range f = y).
\end{theorem}

% Theorem 100
\begin{theorem}[100]
If r wellorders x and s wellorders y and x is a set and y is not a set
then there exists a %unique
 map f such that f orderpreserves r and s in x and y
and \domain f = x.
\end{theorem}

\end{forthel}

\section{ORDINALS}
\begin{forthel}

Let $x, y$ denote classes.
% AXIOM VII: Axiom of regularity
\begin{axiom}
If $x \neq 0$ then there is an element $y$ of $x$ such that y is a class and $x \cap y = 0$.
\end{axiom}

% Theorem 101
\begin{theorem}[101]
$x \notin x$.
\end{theorem}

% Theorem 102
\begin{theorem}[102]
Not ($x \in y$ and $y \in x$).
\end{theorem}

% Definition 103
\begin{definition}[103]
$E = \{[x,y] | x,y$ are classes and $x \in y\}$.
\end{definition}

% Unnumbered
\begin{lemma}
E is a relation.
\end{lemma}

% Unnumbered
\begin{lemma}
If $x \in y$ and $y$ is not a set then $[x,y] = \cal{U}$ and $[x,y] \notin E$.
\end{lemma}

% Theorem 104
\begin{theorem}[104]
$E$ is not a set.
\end{theorem}

% Definition 105
\begin{definition}[105]
A fullclass is a class $x$ such that each element of $x$ is contained in $x$.
\end{definition}

[synonym ordinal/-s]
% Definition 106
\begin{definition}[106]
An ordinal is a class $x$ such that $E$ connects $x$ and $x$ is a fullclass.
\end{definition}

% Theorem 107
\begin{theorem}[107]
If $x$ is an ordinal then $E$ wellorders $x$.
\end{theorem}

% Theorem 108
\begin{theorem}[108]
If $x$ is an ordinal and $y \subset x$ and $y \neq x$ and $y$ is a fullclass then $y \in x$.
\end{theorem}

% Theorem 109
\begin{theorem}[109]
If $x$ is an ordinal and $y$ is an ordinal then $x \subset y$ or $y \subset x$.
\end{theorem}

% Theorem 110
\begin{theorem}[110]
If $x$ is an ordinal and $y$ is an ordinal then $x \in y$ or $y \in x$ or $x = y$.
\end{theorem}

% Theorem 111
\begin{theorem}[111]
If $x$ is an ordinal and $y \in x$ then $y$ is an ordinal.
\end{theorem}

% Definition 112
\begin{definition}
$R = \{$class $x : x$ is an ordinal$\}$.
\end{definition}

% Theorem 113
\begin{theorem}[113]
$R$ is an ordinal and $R$ is not a set.
\end{theorem}

% Theorem 114
\begin{theorem}[114]
Each section of $R$ in $E$ is an ordinal.
\end{theorem}

% Definition 115
\begin{definition}[115]
An ordinalnumber is a class $x$ such that $x \in R$.
\end{definition}

% Definition 116
\begin{definition}[116]
$x < y$ iff $x \in y$.
\end{definition}

% Definition 117
\begin{definition}[117]
$x \leqq y$ iff $x \in y$ or $x = y$.
\end{definition}

% Theorem 118
\begin{theorem}[118]
If $x, y$ are ordinals then $x \leqq y$ iff $x \subset y$.
\end{theorem}

% Theorem 119
\begin{theorem}[119]
If $x$ is an ordinal then $x = \{$ class $y : y \in R$ and $y < x\}$.
\end{theorem}

% Theorem 120
\begin{theorem}[120]
If $x \subset R$ then $\bigcup x$ is an ordinal.
\end{theorem}

% Theorem 121
\begin{theorem}[121]
If $x \subset R$ and $x \neq 0$ then $\bigcap x \in x$.
\end{theorem}

% Definition 122
\begin{definition}[122]
$x + 1 = x \cup <x>$.
\end{definition}

%Let $cy$ denote $\{$ set $y : y \in R$ and $x < y\}$.
% Theorem 123
\begin{theorem}[123]
%Assume that $cy$ is equal to \{$ set $y : y \in R$ and $x < y\}$.
If $x \in R$ then there exists a class z such that z = \{class y : y \in R and x < y\} and
$x + 1$ minimizes z in $E$.
%If $x \in R$ then $x + 1$ minimizes $cy$ in $E$.
\end{theorem}

% Theorem 124
\begin{theorem}[124]
If $x \in R$ then $\bigcup (x + 1) = x$.
\end{theorem}

Let $f$ denote a relation.
% Definition 125
\begin{definition}
$f | x = f \cap (x \times \cal{U})$.
\end{definition}

% Unnumbered
\begin{lemma}
$f | x$ is a relation.
\end{lemma}

Let the restriction of $f$ to $x$ denote $f | x$.

% Theorem 126a
\begin{theorem}[126a]
If $f$ is a map then $f | x$ is a map such that $\domain (f | x) = x \cap (\domain f)$.
%and $(f | x)(y) = f(y)$ for each $y$ such that $y \in \domain (f | x)$).
\end{theorem}

% Theorem 126b
\begin{theorem}[126b]
If $f$ is a map then for each set $y$ if $y \in \domain (f | x)$ then $(f | x)(y) = f(y)$.
\end{theorem}

Let $f, h$ stand for maps.
% Theorem 127
\begin{theorem}[127]
If $f$ is a map such that $\domain f$ is an ordinal and for each $u$ such that $u \in \domain f$ $f(u) = g(f | u)$
and $h$ is a map such that $\domain h$ is an ordinal and for each $v$ such that $v \in \domain h$ $h(u) = g(h | u)$
then $h \subset f$ or $f \subset h$.
\end{theorem}

% Theorem 128
\begin{theorem}[128]
For each $g$ there exists a %unique
map $f$ such that $\domain f$ is an ordinal
and for each ordinalnumber $x$ $f(x) = g(f | x)$.
\end{theorem}

\end{forthel}

\section{INTEGERS}
\begin{forthel}

% Axiom of infinity
\begin{axiom}[VIII]
There exists a set $y$ such that $0 \in y$ and if $x \in y$ then $x \cup <x> \in y$.
\end{axiom}

% Unnumbered
\begin{lemma}
0 is a set.
\end{lemma}

\begin{lemma}
E^{-1} is a relation.
\end{lemma}


[synonym integer/-s]
% Definition 129
\begin{definition}[129]
An integer is an ordinal $x$ such that $E^{-1}$ wellorders $x$.
\end{definition}

% Definition 130
\begin{definition}[130]
$x$ maximizes $y$ in $E$ iff $x$ minimizes $y$ in $E^{-1}$.
\end{definition}

% Definition 131
\begin{definition}[131]
$\omega = \{$ object $x : x$ is an integer $\}.$
\end{definition}

% Theorem 132
%\begin{theorem}[132]
%Each element of an integer is an integer.
%\end{theorem}

% Theorem 133
\begin{theorem}[133]
If $y \in R$ and $x$ maximizes $y$ in $E$ then $y = x + 1$.
\end{theorem}

% Theorem 134
\begin{theorem}[134]
If $x \in \omega$ then $x + 1 \in \omega$.
\end{theorem}

% Theorem 135
\begin{theorem}[135]
$0 \in \omega$ and if $x \in \omega$ then $0 \neq x + 1$.
\end{theorem}

% Theorem 136
\begin{theorem}[136]
If $x, y$ are elements of $\omega$ and $x + 1 = y + 1$ then $x = y$.
\end{theorem}

% Theorem 137
\begin{theorem}[137]
If $x \subset \omega$ and $0 \in x$ and for each $u$ if $u \in x$ then $u + 1 \in x$ then $x = \omega$.
\end{theorem}

% Theorem 138
\begin{theorem}[138]
$\omega \in R$.
\end{theorem}

\end{forthel}

\section{THE CHOICE AXIOM}
\begin{forthel}

% Definition 139
\begin{definition}
A choicefunction is a map $c$ such that for each class $x$ if $x \in \domain c$ then $c(x) \in x$.
\end{definition}

% Axiom of choice
\begin{axiom}[IX]
There is a choicefunction $c$ such that $\domain c = \cal{U} \sim 0$.
\end{axiom}

% Theorem 140
\begin{theorem}[140]
If $x$ is a set then there exists a oneonefunction $f$ such that $\domain f = x$ and $\domain f$ is an ordinalnumber.
\end{theorem}

[synonym nest/-s]
% Definition 141
\begin{definition}[141]
A nest is a set $n$ such that for each set $x, y$ if $x, y \in n$ then $x \subset y$ or $y \subset x$.
\end{definition}

Let $n, m$ denote sets.
% Theorem 142
\begin{theorem}[142]
If $n$ is a nest and each element of $n$ is a nest then $\bigcup n$ is a nest.
\end{theorem}

% Theorem 143
\begin{theorem}[143]
If $x$ is a set then there is a nest $n$ such that $n \subset x$ and (if $m$ is a nest and $m \subset x$ and $n \subset m$ then $m = n$).
\end{theorem}

\end{forthel}

\section{CARDINAL NUMBERS}
\begin{forthel}

% Definition 144
\begin{definition}[144]
Assume that x,y are classes.
$x \approx y$ iff there exists a oneonefunction $f$ such that $\domain f = x$ and $\range f = y$.
\end{definition}

% Theorem 145
\begin{theorem}[145]
$x \approx x$.
\end{theorem}

% Theorem 146
\begin{theorem}[146]
If $x \approx y$ then $y \approx x$.
\end{theorem}

% Theorem 147
\begin{theorem}[147]
If $x \approx y$ and $y \approx z$ then $x \approx z$.
\end{theorem}

% Definition 148
\begin{definition}[148]
A cardinalnumber is an ordinalnumber $x$ such that for each set $y$ if $y \in R$ and $y < x$ then not ($x \approx y$).
\end{definition}

% Definition 149
\begin{definition}[149]
$C = \{$ set $x : x$ is a cardinalnumber $\}$.
\end{definition}

% Theorem 150
\begin{theorem}[150]
$E$ wellorders $C$.
\end{theorem}

% Definition 151
\begin{definition}[151]
$P = \{[x,y] : x,y$ are classes and $x \approx y$ and $y \in C\}$.
\end{definition}

% Theorem 152a
\begin{theorem}[152a]
$P$ is a map.
\end{theorem}

% Theorem 152b
\begin{theorem}[152b]
$\domain P = \cal{U}$.
\end{theorem}

% Theorem 152c
\begin{theorem}[152c]
$\range P = C$.
\end{theorem}

% Theorem 153
\begin{theorem}[153]
If $x$ is a set then there exists a class y such that P(x) = y and $y \approx x$.
\end{theorem}

% Theorem 154
\begin{theorem}[154]
If $x, y$ are sets then $x \approx y$ iff $P(x) = P(y)$.
\end{theorem}

% Theorem 155
\begin{theorem}[155]
$P(P(x)) = P(x)$.
\end{theorem}

% Theorem 156
\begin{theorem}[156]
$x \in C$ iff $x$ is a set and $P(x) = x$.
\end{theorem}

% Theorem 157
\begin{theorem}[157]
If $y \in R$ and $x \subset y$ then $P(x) \leqq y$.
\end{theorem}

% Theorem 158
\begin{theorem}[158]
If $y$ is a set and $x \subset y$ then $P(x) \leqq P(y)$.
\end{theorem}

% Theorem 159
\begin{theorem}[159]
If $x, y, u, v$ are sets and $u \subset x$ and $v \subset y$ and $x \approx v$ and $y \approx u$ then $x \approx y$.
\end{theorem}

% Theorem 160
\begin{theorem}[160]
If $f$ is a map and $f$ is a set then $P(\range f) \leqq P(\domain f)$.
\end{theorem}

% Theorem 161
\begin{theorem}[161]
If $x$ is a set then $P(x) < P(2^{x})$.
\end{theorem}

% Theorem 162
\begin{theorem}[162]
$C$ is not a set.
\end{theorem}

% Theorem 163
\begin{theorem}[163]
If $x \in \omega$ and $y \in \omega$ and $x + 1 \approx y + 1$ then $x \approx y$.
\end{theorem}

% Theorem 164
\begin{theorem}
$\omega \subset C$.
\end{theorem}

% Theorem 165
\begin{theorem}[165]
$\omega \in C$.
\end{theorem}

Let x denote a class.
% Definition 166
\begin{definition}[166]
$x$ is finite iff $P(x) \in \omega$.
\end{definition}

% Theorem 167
\begin{theorem}[167]
$x$ is finite iff there is a relation $r$ such that $r$ wellorders $x$ and $r^{-1}$ wellorders $x$.
\end{theorem}

% Theorem 168
\begin{theorem}[168]
If $x, y$ are finite then $x \cup y$ is finite.
\end{theorem}

% Theorem 169
\begin{theorem}[169]
If $x$ is finite and for each class y if $y \in x$ then y is finite then $\bigcup x$ is finite.
\end{theorem}

% Theorem 170
\begin{theorem}[170]
If $x, y$ are finite then $x \times y$ is finite.
\end{theorem}

% Theorem 171
\begin{theorem}[171]
If $x$ is finite then $2^{x}$ is finite.
\end{theorem}

% Theorem 172
\begin{theorem}[172]
If $x$ is finite and $y \subset x$ and $P(y) = P(x)$ then $x = y$.
\end{theorem}

% Theorem 173
\begin{theorem}[173]
If $x$ is a set and $x$ is not finite then there is a set $y$ such that $y \subset x$ and $y \neq x$ and $x \approx y$.
\end{theorem}

% Theorem 174
\begin{theorem}[174]
If $x \in R \sim \omega$ then $P(x + 1) = P(x)$.
\end{theorem}

% Definition 175
\begin{definition}[175]
$\max[x,y] = x \cup y$.
\end{definition}

% Assisting definition
\begin{definition}
$[u,v] \ll [x,y]$ iff [u,v] \in R \times R and [x,y] \in R \times R and 
($\max[u,v] < \max[x,y]$ or ($\max[u,v] = \max[x,y]$ and $u < x$)
	or ($\max[u,v] = \max[x,y]$ and $u = x$ and $v < y$)).
\end{definition}

% Assisting
\begin{definition}
$\lll = \{$class $z :$ for some classes $u,v,x,y,a,b$ such that
	$[u,v] \ll [x,y]$ and a = [u,v] and b = [x,y] $z = [a,b]\}$.
\end{definition}

% Assisting
\begin{lemma}
$\lll$ is a relation.
\end{lemma}

% Definition 176
%\begin{definition}[176]
%$\ll = \{$set $z : z$ is an ordinal\}$. %and $ ([u,v]) \in (R \times R)$ and $([x,y]) \in (R \times R)$ and ($\max[u,v] < \max[x,y]$
%or ($\max[u,v] = \max[x,y]$ and $u < x$) or ($\max[u,v] = \max[x,y]$ and $u = x$ and $v < y))\}$.
%\end{definition}
%\begin{definition}[176]
%	$\ll = \{u, v, x, y:$ for some $u, v, x, y$ such that $[u,v] \in (R \times R)$ and
%	$[x,y] \in (R \times R)$ and $u$ is semismall in $v$ to $x$ and $y$\}$.
%\end{definition}

% Theorem 177
\begin{theorem}[177]
$\lll$ wellorders $R \times R$.
\end{theorem}

% Theorem 178
\begin{theorem}[178]
If $[u,v] \ll [x,y]$ then $[u,v] \in ((\max[x,y] + 1) \times (\max[x,y] + 1))$.
\end{theorem}

% Theorem 179
\begin{theorem}[179]
If $x \in C \sim \omega$ then $P(x \times x) = x$.
\end{theorem}

% Theorem 180
\begin{theorem}[180]
If $x, y$ are elements of $C$ and ($x \notin \omega$ or $y \notin \omega$) then $P(x \times y) = \max[P(x),P(y)]$.
\end{theorem}

% Assisting
\begin{definition}
$< = \{$object $z :$ for some sets $x,y$ $(x < y$ and $z = [x,y])\}$.
\end{definition}

% Theorem 181
\begin{theorem}
There is a %unique
map $f$ such that $f$ orderpreserves $<$ and $<$ and $\domain f = R$ and $\range f = C \sim \omega$.
\end{theorem}

\end{forthel}

\end{document}
