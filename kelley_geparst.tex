% Kelley Morse Theory of Sets and Classes

% We formalize the 

% Appendix: ELEMENTARY SET THEORY 

% of John L. Kelley 
% GENERAL TOPOLOGY 
% D. Van Nostrand Company Inc. 1955
%
% The appendix develops what is known as Kelley-Morse
% class theory (KM). 
% Kelley writes: "The system of axioms adopted is a variant
% of systems of Skolem and of A.P.Morse and owes much to
% the Hilbert-Bernays-von Neumann system as formulated
% by Gödel."

% This file covers the first 56 top level sections of the appendix.
% It uses SADs inbuilt class notion and mechanisms to model the
% classes of Kelley. We have built the class notion into 
% Naproche-SAD by replacing "set" by "class"

% This file checks in ~ 1 min on my laptop.

\documentclass[a4paper,draft]{amsproc}
\title{\textbf{Kelley}}

\date{}
\begin{document}

\theoremstyle{plain}
 \newtheorem{theorem}{Theorem}
 \newtheorem*{theorem*}{Theorem}
 \newtheorem{lemma}[theorem]{Theorem}
 \newtheorem*{lemma*}{Theorem}
 \newtheorem{proposition}[theorem]{Theorem}
 \newtheorem*{proposition*}{Theorem}
\theoremstyle{definition}
 \newtheorem{example}[theorem]{Example}
 \newtheorem*{example*}{Example}
 \newtheorem{definition}[theorem]{Definition}
 \newtheorem*{definition*}{Definition}
 \newtheorem{signature}[theorem]{Signature}
 \newtheorem*{signature*}{Signature}
\theoremstyle{remark}
 \newtheorem{remark[theorem]}{Remark}
 \newtheorem*{remark*}{Remark}
 \newtheorem{notation}[theorem]{Notation}
 \newtheorem*{notation*}{Notation}
\theoremstyle{axiom}
 \newtheorem{axiom}{Axiom}
 \newtheorem*{axiom*}{Axiom}
 \numberwithin{equation}{section}

\newenvironment{forthel}{}{}
\maketitle

\newcommand{\cal}[1]{\mathcal{#1}}
\newcommand{\domain}[1]{\textnormal{domain}}
\newcommand{\range}[1]{\textnormal{range}}
\newcommand{xxx}{\} % geht das?

\begin{forthel}
[synonym element/-s]

\begin{signature}[ElmSort]
An element is a notion.
\end{signature}

% section not yet supported?
%\section{The Classification Axiom Scheme}

Let x, y, z, u, v, a, b, c, d, e stand for classes.

Let a xxxneq b stand for a != b.
Let a xxxin b stand for a is an element of b.

% Axiom I. Axiom of extent.
\begin{axiom}[I] 
For each x, y  
x = y if and only if for each z z xxxin x 
when and only when z xxxin y.
\end{axiom}

% II Classification axiom-scheme corresponds to the way
% "classifications", i.e., abstraction terms are handled
% in Naproche-SAD

[synonym set/-s]

% Definition 1
\begin{definition}[1]
a set is a class x such that for some class y x xxxin y.
%x is a set iff for some y x xxxin y.
\end{definition}

%\section{ELEMENTARY ALGEBRA OF CLASSES}

% Definition 2
\begin{definition}[2] 
x xxxcup y = {set z : z xxxin x or z xxxin y }.
\end{definition}

% Definition 3
\begin{definition}[3] 
x xxxcap y = {set u : u xxxin x and u xxxin y }.
\end{definition}

Let the union of x and y stand for x xxxcup y.
Let the intersection of x and y stand for x xxxcap y.

% Theorem 4
\begin{theorem}[4]
(z xxxin x xxxcup y iff z xxxin x or z xxxin y)
and (z xxxin x xxxcap y iff z xxxin x and z xxxin y).
\end{theorem}

% Theorem 5
\begin{theorem}[5]
x xxxcup x = x and x xxxcap x = x.
\end{theorem}

% Theorem 6
\begin{theorem}[6]
x xxxcup y = y xxxcup x and x xxxcap y = y xxxcap x.
\end{theorem}

% Theorem 7
\begin{theorem}[7]
(x xxxcup y) xxxcup z = x xxxcup (y xxxcup z) 
and (x xxxcap y) xxxcap z = x xxxcap (y xxxcap z).
\end{theorem}

% Theorem 8
\begin{theorem}[8]
x xxxcap (y xxxcup z) = (x xxxcap y) xxxcup (x xxxcap z)
and x xxxcup (y xxxcap z) = (x xxxcup y) xxxcap (x xxxcup z).
\end{theorem}

% 9 Definition, as a parser directive.
% Definition 9
Let a xxxnotin b stand for a is not an element of b.

% Definition 10
\begin{definition}[10] 
xxxsim x = {set y : y xxxnotin x}.\end{definition}
Let the complement of x stand for xxxsim x.

% Theorem 11
\begin{theorem}[11]
xxxsim (xxxsim x) = x.
\end{theorem}

% Theorem 12 (De Morgan)
\begin{theorem}[12]
xxxsim (x xxxcup y) = (xxxsim x) xxxcap (xxxsim y) 
and xxxsim (x xxxcap y) = (xxxsim x) xxxcup (xxxsim y).
\end{theorem}

% Theorem 13
\begin{definition}[13] x xxxsim y = x xxxcap (xxxsim y).\end{definition}

% Theorem 14
\begin{theorem}[14]
x xxxcap (y xxxsim z) = (x xxxcap y) xxxsim z.
\end{theorem}

% Definition 15
\begin{definition}[15] 
0 = {set x : x xxxneq x}.
\end{definition}
Let the void class stand for 0.
Let zero stand for 0.

% Theorem 16
\begin{theorem}[16]
x xxxnotin 0.
\end{theorem}

% Theorem 17
\begin{theorem}[17]
0 xxxcup x = x and 0 xxxcap x = 0.
\end{theorem}

% Definition 18
\begin{definition}[18]
	xxxcal{U} = {set x : x = x}.
\end{definition}
Let the universe stand for xxxcal{U}.

% Theorem 19
\begin{theorem}[19]
x xxxin xxxcal{U} if and only if x is a set.
\end{theorem}

% Theorem 20
\begin{theorem}[20]
x xxxcup xxxcal{U} = xxxcal{U} and x xxxcap xxxcal{U} = x.
\end{theorem}

% Theorem 21
\begin{theorem}[21]
xxxsim 0 = xxxcal{U} and xxxsim xxxcal{U} = 0.
\end{theorem}

% Definition 22
\begin{definition}[22] xxxbigcap x = 
{set z : for each y if y xxxin x then z xxxin y}.\end{definition}

% Definition 23
\begin{definition}[23] xxxbigcup x = 
{set z : for some y (z xxxin y and y xxxin x)}.\end{definition}

Let the intersection of x stand for xxxbigcap x.
Let the union of x stand for xxxbigcup x.

% Theorem 24
\begin{theorem}[24]
xxxbigcap 0 = xxxcal{U} and xxxbigcup 0 = 0.
\end{theorem}

% Definition 25
\begin{definition}[25]
a subclass of y is a class x such that 
%x is a subclass of y iff 
for each class z if z xxxin x then z xxxin y.
\end{definition}

Let x xxxsubset y stand for x is a subclass of y.
Let x is contained xxxin y stand for x xxxsubset y.

% Unnumbered
\begin{lemma}
0 xxxsubset 0 and 0 xxxnotin 0.
\end{lemma}

% Theorem 26
\begin{theorem}[26]
0 xxxsubset x and x xxxsubset xxxcal{U}.
\end{theorem}

% Theorem 27
\begin{theorem}[27]
x = y iff x xxxsubset y and y xxxsubset x.
\end{theorem}

% Theorem 28
\begin{theorem}[28]
If x xxxsubset y and y xxxsubset z then x xxxsubset z.
\end{theorem}

% Theorem 29
\begin{theorem}[29]
x xxxsubset y iff x xxxcup y = y.
\end{theorem}

% Theorem 30
\begin{theorem}[30]
x xxxsubset y iff x xxxcap y = x.
\end{theorem}

% Theorem 31
\begin{theorem}[31]
If x xxxsubset y then xxxbigcup x xxxsubset xxxbigcup y
and xxxbigcap y xxxsubset xxxbigcap x.
\end{theorem}

% Theorem 32
\begin{theorem}[32]
If x xxxin y then x xxxsubset xxxbigcup y 
and xxxbigcap y xxxsubset x.
\end{theorem}

%\section{EXISTENCE OF SETS}

% Axiom of subsets.
\begin{axiom}
If x is a set then there is a set y such that for each
z if z xxxsubset x then z xxxin y.
\end{axiom}

% This axiom is a kind of powerclass axiom, where the powerclass
% also has all subCLASSES as elements.

% Theorem 33
\begin{theorem}[33]
If x is a set and z xxxsubset x then z is a set.
\end{theorem}

% Theorem 34
\begin{theorem}[34]
0 = xxxbigcap xxxcal{U} and xxxcal{U} = xxxbigcup xxxcal{U}.
\end{theorem}

% Theorem 35
\begin{theorem}[35]
If x xxxneq 0 then xxxbigcap x is a set.
\end{theorem}

% Definition 36
\begin{definition}[36] 2^{x} = {set y : y xxxsubset x}.\end{definition}

% Theorem 37
\begin{theorem}[37]
xxxcal{U} = 2^{xxxcal{U}}.
\end{theorem}

% Theorem 38
%[prove off]
\begin{theorem}[38]
If x is a set then 2^{x} is a set and for
each y  y xxxsubset x iff y xxxin 2^{x}.
Proof. Let x be a set.
Take a set y such that for each z 
if z xxxsubset x then z xxxin y.
2^{x} xxxsubset y.
qed.
\end{theorem}
%[/prove]

% The Russell paradox.
% Unnumbered
\begin{definition} xxxcal{R} = {set x : x xxxnotin x}.\end{definition}

% Unnumbered
\begin{theorem}
xxxcal{R} is not a set.
\end{theorem}

% Theorem 39
\begin{theorem}[39]
xxxcal{U} is not a set.
\end{theorem}

% Definition 40
\begin{definition}[40] <x> = {set z : if x xxxin xxxcal{U} then z = x}.\end{definition}
Let the singleton of x stand for <x>.

% Before We used <x> instead of {x} since {x} was an inbuilt 
% set notation   

% Theorem 41
\begin{theorem}[41]
If x is a set then for each y y xxxin <x> iff y = x.
\end{theorem}

% Theorem 42
\begin{theorem}[42]
If x is a set then <x> is a set.
Proof. Let x be a set. Then <x> xxxsubset 2^{x} 
and 2^{x} is a set.
qed.
\end{theorem}

% Theorem 43
\begin{theorem}[43]
<x> = xxxcal{U} if and only if x is not a set.
\end{theorem}

% Theorem 44a
\begin{theorem}[44a]
If x is a set then xxxbigcap <x> = x 
and xxxbigcup <x> = x.
\end{theorem}

% Theorem 44b
\begin{theorem}[44b]
If x is not a set then xxxbigcap <x> = 0
and xxxbigcup <x> = xxxcal{U}.
\end{theorem}

% Axiom IV. 
\begin{axiom}[IV]
If x is a set and y is a set then x xxxcup y is a set.
\end{axiom}

% Definition 45
\begin{definition}[45] {x,y} = <x> xxxcup <y>.\end{definition}
Let the unordered pair of x and y stand for {x,y}.

% The following has been a problem before:
% We use <x,y> instead of {x y} because Naproche-SAD requires
% some symbolic or textual material between the variables
% x and y. We use {x;y} instead of {x,y} because the latter
% notion is an inbuilt set notation of Naproche-SAD.

% Theorem 46a
\begin{theorem}[46a]
If x is a set and y is a set 
then {x,y} is a set and (z xxxin {x,y} iff z=x or z=y). 
\end{theorem}

% Theorem 46b
\begin{theorem}[46b]
({x,y}) = xxxcal{U} if and only if (x is not a set or y is not a set).
\end{theorem}

% Theorem 47a
\begin{theorem}[47a]
If x,y are sets then xxxbigcap {x,y} = x xxxcap y
and xxxbigcup {x,y} = x xxxcup y.
Proof. Let x,y be sets.
xxxbigcup {x,y} xxxsubset x xxxcup y.
x xxxcup y xxxsubset xxxbigcup {x,y}.qed.
\end{theorem}

% Theorem 47b
\begin{theorem}[47b]
If x is not a set or y is not a set then
xxxbigcap {x,y} = 0 and xxxbigcup {x,y} = xxxcal{U}.
\end{theorem}

%\section{ORDERED PAIRS: RELATIONS}

% Definition 48
\begin{definition}[48] [x,y] = {<x>,{x,y}}.\end{definition}
Let the ordered pair of x and y stand for [x,y].

% Theorem 49a
\begin{theorem}[49a]
[x,y] is a set if and only if x is a set and y is a set.
\end{theorem}

% Theorem 49b
\begin{theorem}[49b]
If [x,y] is not a set then [x,y] = xxxcal{U}.
\end{theorem}

% Theorem 50a
\begin{theorem}[50a]
If x and y are sets then 
  (xxxbigcup [x,y]) = ({x,y}) and
  (xxxbigcap [x,y]) = <x> and
  (xxxbigcup xxxbigcap [x,y]) = x and
  (xxxbigcap xxxbigcap [x,y]) = x and
  (xxxbigcup xxxbigcup [x,y]) = x xxxcup y and
  (xxxbigcap xxxbigcup [x,y]) = x xxxcap y.
\end{theorem}

% Theorem 50b
\begin{theorem}[50b]
If x is not a set or y is not a set then
  xxxbigcup xxxbigcap [x,y] = 0 and
  xxxbigcap xxxbigcap [x,y] = xxxcal{U} and
  xxxbigcup xxxbigcup [x,y] = xxxcal{U} and
  xxxbigcap xxxbigcup [x,y] = 0.
\end{theorem}

% Definition 51
\begin{definition}[51] 1^{st} coord z = xxxbigcap xxxbigcap z.\end{definition}

% Definition 52
\begin{definition}[52] 2^{nd} coord z = (xxxbigcap xxxbigcup z) xxxcup 
((xxxbigcup xxxbigcup z) xxxsim xxxbigcup xxxbigcap z).\end{definition} 
Let the first coordinate of z stand for 1^{st} coord z.
Let the second coordinate of z stand for 2^{nd} coord z.

% Theorem 53
\begin{theorem}[53]
2^{nd} coord xxxcal{U} = xxxcal{U}.
\end{theorem}

% Theorem 54a
\begin{theorem}[54a]
If x and y are sets 
then 1^{st} coord [x,y] = x and 2^{nd} coord [x,y] = y.
Proof. Let x and y be sets.
2^{nd} coord [x,y] = (xxxbigcap xxxbigcup [x,y]) xxxcup 
((xxxbigcup xxxbigcup [x,y]) xxxsim xxxbigcup xxxbigcap [x,y])
= (x xxxcap y) xxxcup ((x xxxcup y) xxxsim x)
= y.qed.
\end{theorem}

% Theorem 54b
\begin{theorem}[54b]
If x is not a set or y is not a set then
1^{st} coord [x,y] = xxxcal{U} and 
2^{nd} coord [x,y] = xxxcal{U}.
\end{theorem}

% Theorem 55
\begin{theorem}[55]
If x and y are sets and [x,y] = [u,v] then
x = u and y = v.
\end{theorem}

% We can interpret xxxcal{U} to mean undefined.
% Then ( , ) produces a a set or undefined.
% We can instead extend the signature (our language)
% by an elementary symbol ( , ), satisfying standard axioms ... .
% Ideally, we would like ( , ) to be an "object" and
% not a set. Sets will also be objects.

[synonym relation/-s]

% Definition 56
\begin{definition}[56] 
A relation is a class r such that 
%A class r is a relation if and only if 
for each element z of r
there is x and y such that z = [x,y].
\end{definition}

Let r, s, t stand for relations.

% Definition 57
\begin{definition}[57]
r xxxcirc s = {[x,z] | x, z are objects and there exists an object b 
such that [x,b] xxxin s and [b,z] xxxin r}. 
\end{definition}

% Theorem 58
\begin{theorem}[58]
(r xxxcirc s) xxxcirc t = r xxxcirc (s xxxcirc t).
Proof. (r xxxcirc s) xxxcirc t xxxsubset r xxxcirc (s xxxcirc t) and
r xxxcirc (s xxxcirc t) xxxsubset (r xxxcirc s) xxxcirc t.qed.
\end{theorem}

% Theorem 59a
\begin{theorem}[59a]
r xxxcirc (s xxxcup t) = (r xxxcirc s) xxxcup (r xxxcirc t).
Proof.
r xxxcirc (s xxxcup t) xxxsubset (r xxxcirc s) xxxcup (r xxxcirc t).
(r xxxcirc s) xxxcup (r xxxcirc t) xxxsubset r xxxcirc (s xxxcup t).qed.
\end{theorem}

% Theorem 59b
\begin{theorem}[59b]
r xxxcirc (s xxxcap t) xxxsubset (r xxxcirc s) xxxcap (r xxxcirc t).
\end{theorem}

% Definition 60
\begin{definition}[60]
r^{-1} = {[b,a] | a, b are objects and [a,b] xxxin r}.
\end{definition}
Let the relation inverse to r stand for r^{-1}.

% Unnumbered
\begin{lemma}
r^{-1} is a relation.
\end{lemma}

% Theorem 61
\begin{theorem}[61]
(r^{-1})^{-1} = r.
Proof. r xxxsubset (r^{-1})^{-1}.
(r^{-1})^{-1} xxxsubset r.qed.
\end{theorem}

% Lemma 62a
\begin{lemma}[62a]
Assume r xxxsubset s. Then r^{-1} xxxsubset s^{-1}.
\end{lemma}

%[/prove]

% Lemma 62b
\begin{lemma}[62b]
(r xxxcirc s)^{-1} xxxsubset (s^{-1}) xxxcirc (r^{-1}).
Proof. Let u xxxin (r xxxcirc s)^{-1}.
Take c and a such that u = [c,a].
Take an object b such that ([a,b] xxxin s and [b,c] xxxin r).
Indeed [a,c] xxxin r xxxcirc s.
[b,a] xxxin s^{-1} and [c,b] xxxin r^{-1}.
Then [c,a] xxxin (s^{-1}) xxxcirc (r^{-1}).qed.
\end{lemma}

% Unnumbered
\begin{lemma}
(s^{-1}) xxxcirc (r^{-1}) xxxsubset (r xxxcirc s)^{-1}.
Proof.
((s^{-1}) xxxcirc (r^{-1}))^{-1} xxxsubset ((r^{-1})^{-1}) xxxcirc ((s^{-1})^{-1}) (by 62b).
((s^{-1}) xxxcirc (r^{-1}))^{-1} xxxsubset r xxxcirc s (by 61).
(((s^{-1}) xxxcirc (r^{-1}))^{-1})^{-1} xxxsubset (r xxxcirc s)^{-1} (by 62a).
(s^{-1}) xxxcirc (r^{-1}) xxxsubset (r xxxcirc s)^{-1} (by 61).qed.
\end{lemma}

% Theorem 62
\begin{theorem}[62]
(r xxxcirc s)^{-1} = (s^{-1}) xxxcirc (r^{-1}).
Proof. (r xxxcirc s)^{-1} xxxsubset (s^{-1}) xxxcirc (r^{-1}).
(s^{-1}) xxxcirc (r^{-1}) xxxsubset (r xxxcirc s)^{-1}.qed.
\end{theorem}

% Functions

% Since "function" is predefined in SAD3, we use the word "map" instead.

%[/prove]

[synonym map/-s]
% Definition 63
\begin{definition}[63]
A map is a relation f such that for each a, b, c
if [a,b] xxxin f and [a,c] xxxin f then b = c.
\end{definition}

Let f, g stand for maps.

% Theorem 64
\begin{theorem}[64]
If f, g are maps then f xxxcirc g is a map.
\end{theorem}

% Definition 65
\begin{definition}[65]
xxxdomain x = {object u |  there exists an object v such that [u,v] xxxin x}.
\end{definition}

% Definition 66
\begin{definition}[66]
xxxrange x = {object v |  there exists an object u such that [u,v] xxxin x}.
\end{definition}

% Theorem 67
%\begin{theorem}[67]
%xxxdomain xxxcal{U} = xxxcal{U} and xxxrange xxxcal{U} = xxxcal{U}.
%\end{theorem}
%\begin{proof}
%If x xxxin xxxcal{U} then (x,0), (0,x) xxxin xxxcal{U}.
%x xxxin xxxdomain xxxcal{U} and x xxxin xxxrange xxxcal{U}.
%\end{proof}

% Signature 68
\begin{signature}[68]
Let f be a map. Let u xxxin xxxdomain f.
The value of f at u is an object v such that [u,v] xxxin f.
\end{signature}
Let f(u) stand for the value of f at u.


% Theorem 69
\begin{theorem}[69a]
If x xxxnotin xxxdomain f then f(x) = xxxcal{U}.
%Proof. 
%If x xxxnotin xxxdomain f 
%then {class y: [x,y] xxxin f} = 0 and f(x) = xxxcal{U} (by 24).
%qed.
\end{theorem}

\begin{theorem}[69b]
If x xxxin xxxdomain f then f(x) xxxin xxxcal{U}.
%Proof.
%If x xxxin xxxdomain f 
%then {class y : [x,y] xxxin f} xxxneq 0
%and f(x) is a set (by 35).qed.
\end{theorem}

% Theorem 70
\begin{theorem}[70]
Let f be a map. Then f = {[x,y] : y = f(x)}.
%Then f = {[u,f(u)] : u xxxin xxxdomain f}.
\end{theorem}

% Theorem 71
\begin{theorem}[71]
Assume xxxdomain f = xxxdomain g and for all elements u of xxxdomain f
%u xxxin xxxdomain f 
f(u) = g(u). Then f = g.
Proof. Let us show that f xxxsubset g.
Let w xxxin f. 
Then w xxxin g. end.
Let us show that g xxxsubset f.
Let w xxxin g.  
Take objects u, v such that w=[u,v].
u xxxin xxxdomain g and v = g(u).
Then u xxxin xxxdomain f and v = f(u).
Then w xxxin f. end.
qed.
\end{theorem}

% Axiom of substitution
\begin{axiom}[V]
Let f be a map. If xxxdomain f is a set then xxxrange f is a set.
\end{axiom}

% Axiom of amalgamation
\begin{axiom}[VI]
If x is a set then xxxbigcup x is a set.
\end{axiom}

\begin{definition}[72]
 x xxxtimes y = {[u,v] : u xxxin x and v xxxin y}.
\end{definition}

\begin {theorem}[73]
If u,v are sets then <u> xxxtimes <v> is a set.
\end{theorem}

\begin{theorem}[74]
If x,y are sets then x xxxtimes y is a set.
Proof.
Let f be a map such that xxxdomain f = x and
f(u) = <u> xxxtimes y for every element u of x.
xxxrange f is a set.
xxxrange f = {class z : for some class u (u xxxin x 
and z = (<u> xxxtimes y))}.
Consequently xxxbigcup xxxrange f = x xxxtimes y.
qed.
\end{theorem}

\begin{theorem}[75]
If f is a map and xxxdomain f is a set 
then f is a set.
Proof. f xxxsubset (xxxdomain f xxxtimes xxxrange f).
qed.
\end{theorem}

\begin{definition}[76]
y^{x} = {map f : xxxdomain f = x and xxxrange f xxxsubset y}.
\end{definition}

\begin{theorem}[77]
If x,y are sets then y^{x} is a set.
Proof. 
Let f be map.
If f xxxin y^{x} then f xxxsubset x xxxtimes y.
f xxxin 2^{x xxxtimes y} (by 38).
2^{x xxxtimes y} is a set.
y^{x} xxxsubset 2^{x xxxtimes y}.
y^{x} is a set.qed.
\end{theorem}

\begin{definition}[78]
f is on x if and only if f is a map and x = xxxdomain f.
\end{definition}

\begin{definition}[79]
f is to y if and only if f is a map such and xxxrange f xxxsubset y.
\end{definition}

\begin{definition}[80]
f is onto y if and only if f is a map and xxxrange f = y.
\end{definition}

\end{forthel}

\end{document}